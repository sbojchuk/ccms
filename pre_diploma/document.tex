%%% initial
\documentclass[ukrainian,14pt,utf8,
%pointsection, % це хз для чого
nocolumnsxix,
nocolumnxxxi,
nocolumnxxxii,
floatsection, %підпис плаваючих об'єктів згдіно номера розліду (таблиці, рисунки)
hpadding=5mm, %відстань до рамки зліва, справа
vpadding=5mm %відстань до рамки зліва, справа
]{eskdtext}

%%% for working with math
%\usepackage{amsmath}
%\usepackage{amssymb}

%%% for working some LaTeX packages
\usepackage{xecyr}

%%% for fonts
\usepackage{xltxtra}
%%% Times New Roman - main font
\setmainfont[Mapping=tex-text]{Times New Roman}
%%% Courier New - fot math
\setmonofont[Scale=MatchLowercase]{Courier New}

%%% for ---, --, << >> и т.п.
\defaultfontfeatures{Mapping=tex-text}

%%% ukrainian text
\usepackage{polyglossia}
\setdefaultlanguage{ukrainian}
\newfontfamily\ukrainianfont{Times New Roman}

%%% for words wrapping
%\XeTeXinterchartokenstate=1
%\XeTeXcharclass `\- 24
%\XeTeXinterchartoks 24 0 ={\hskip\z@skip}
%\XeTeXinterchartoks 0 24 ={\nobreak}

% для абзацу %
\usepackage{indentfirst}

% code highlighting %
\usepackage{listings}
\lstset{
language=Java,
basicstyle=\footnotesize\sffamily,
numbers=left,
numberstyle=\tiny,
frame=tb,
columns=fullflexible,
showstringspaces=false,
breaklines=true 
}

% automaticaly wrap to other line
\sloppy
%\usepackage{setspace}
%\onehalfspacing
\linespread{1.5}

%%% for graphix
\usepackage{graphicx}
\graphicspath{{images/}} %path to images
%%% titile for pictures
\addto{\captionsukrainian}{\renewcommand{\figurename}{Рисунок}}


%%% for internet ulrs
\usepackage{url}

%%% subtitiles(\subsubsection) don't show in article
%\setcounter{tocdepth}{2}

%%% new section from new page
\let\stdsection\section
\renewcommand\section{\newpage\stdsection}

%%% плаваючі обєкти підпис
%\renewcommand\section{\newpage\stdsection}

%%% Numering subtopics begin with B (B.1)
%\makeatletter
%\renewcommand\thesubsection{\ifnum\c@section=0{В.\arabic{subsection}}\else{\arabic{section}.\arabic{subsection}}\fi}
%\makeatother

%%% for coloring rows
\usepackage[table]{xcolor}

%%% для розриву таблиць 
\usepackage{longtable}
%\usepackage{eskdgraph}






\ESKDtitle{ Аналіз розробки програмного алгоритмічного забезпечення багатофункціональної корпоративної системи для сумісної роботи, управління документами і проектами }
\ESKDdocName{\uppercase{Звіт з переддипломної практики}}
\ESKDgroup{ІНФТУНГ ПЗ-07-1}

\ESKDauthor{ Бойчук Я.В. }
\ESKDchecker{ Бандура В.В. }
\ESKDsignature{ ПП.ПЗ 03.00.00.000 ПЗ }

%%finixg ukrainian languale%%
\renewcommand{\ESKDcolumnXfIname}{Розробив}
\renewcommand{\ESKDcolumnXfIIname}{Перевірив}
\renewcommand{\ESKDcolumnXfIVname}{Т.контр}
\renewcommand{\ESKDcolumnXfVname}{Н.контр}
\renewcommand{\ESKDcolumnXfVIname}{Затв.}

\renewcommand{\ESKDcolumnIVname}{Літ.}
\renewcommand{\ESKDcolumnVIIIname}{Аркушів}
\renewcommand{\ESKDcolumnXVIIname}{Підпис}
\renewcommand{\ESKDcolumnXVname}{Арк.}

%%fix ESKDTitle font
%%FIXME change for row, now for font
\renewcommand{\ESKDfontVsize}{%
\ESKDfontSetBaseLineStretch
\fontsize{10pt}{8pt}\selectfont\ESKDfontShape}

%%space after text
\setlength{\ESKDsectionSkipBefore}{-7mm \@plus -3mm \@minus -2mm}
\setlength{\ESKDsectionSkipAfter}{7mm \@plus 1mm \@minus 2mm}
\setlength{\ESKDsubsectionSkipBefore}{-7mm \@plus -3mm \@minus -2mm}
\setlength{\ESKDsubsectionSkipAfter}{7mm \@plus 1mm \@minus 2mm}
\setlength{\ESKDsubsubsectionSkipBefore}{-7mm \@plus -3mm \@minus -2mm}
\setlength{\ESKDsubsubsectionSkipAfter}{7mm \@plus 1mm \@minus 2mm}

%%font size for section (&&sub)
\renewcommand{\ESKDsectionStyle}{\centering\normalfont\bfseries\MakeUppercase}
\renewcommand{\ESKDsubsectionStyle}{\centering\normalfont\bfseries}
\renewcommand{\ESKDsubsubsectionStyle}{\centering\normalfont\bfseries}




        



\begin{document}
\fontsize{14}{14pt}\selectfont

\maketitle


\tableofcontents


\section*{Вступ}
\addcontentsline{toc}{section}{Вступ}
Текст вступу

\section{Заголовок}
Здесь основной текст работы\cite{BookRef, ArticleRef, LinkRef}.


\section{Вступ}
В наш час алгоритми зайняли одну із важливіших ніш нашого життя. Вони застосовуються всюди, починаючи від звичайних калькуляторів, і закінчуючи мільйонами серверів корпорації Google і  Facebook. Алгоритми дають змогу економити мільярди доларів на устаткування і на робочій силі. Тому можна сміло стверджувати, що алгоритми --- це основа програмування.
\par Проте дуже важко стати спеціалістом із алгоритмізація без спеціальних автоматизованих систем, які особливо досить широко використовуються у вищих навчальних закладах. Для прикладу це ejudge, PCMS2, Contester, Executor, PC2. Проте в них є свої недоліки. Одні із них платні, інші не мають відповідної підтримки. Тому було вирішено розробити аналог таких систем для автоматизованого тестування.

\subsection{Постановка задачі}
Метою даної роботи є розробка програмного продукту, який би міг в автоматичному режимі тестувати користувачів. Мається на увазі те, що любий, для прикладу студент, користувач буде мати можливість тестуватися у даній системі. Це буде відбуватися наступний чином: користувач заходить на спеціальний сайт, вибирає задачу, завантажує умову задачі, ознайомлюється із технічними параметрами (ліміт на час виконання, ліміт на пам'ять, тип вхідних і вихідних даних, тощо...) і представляє системі розв'язок у виді вихідних кодів. При цьому у системі повинні бути закладені тести із розв'язком, і система їх повинна автоматично звірити із тестами, які будуть згенеровані під час виконання програми користувача. При цьому потрібно передбачити всі варіанти помилок (помилка компіляції, помилкова відповідь, перевищено ліміт на виконання і т.д.). 
\par Це все повинно буде відбуватися через веб інтерфейс. Щоб не бути залежним від платформи, яку використовує користувач та щоб завжди для користувача була доступна найновіша версія продукту.



\section{Технічний огляд програмного продукту}
Перш за все потрібно було вирішити за допомогою яких технологій буде реалізована дана система. Я поділив систему на дві категорії. Перша - це те що представляється для користувача. В моєму випадку - це інтерактивний веб сайт, де доступна можливість перегляду новин (адміністратору і додовання), завантаження задач, відправлення задач на перевірку. Реалізована система коментування.
\par А серверна частина буде відповідати за отримання задачі, її запуск в певному chroot середовищі (з міркувань безпеки сервера) з подальшою компіляцією задачі і перевіркою на правильність проходження тестів.

\subsection{Веб інтерфейс користувача}
Для створення веб сайту я використав ряд технологій. Сюди входять: PHP, MySQL, HTML, CSS. 
\par Для початку я створив макет сайту у растровому редакторі GIMP. 
\par Я приблизно намалював всі майбутні блоки, які будуть використовуватися на сайті. Потім цей шаблон я зверстав використовуючи мову розмітки HTML і каскадну таблицю стилів CSS. Але поки сайт був статичний. Для <<чистоти есперименту>> я перевірив зверстаний макет на w3c валідаторі \cite{w3c}. Всі тести було пройдено успішно, при чому навіть без попереджень (warning).
\par Далі я приступив до розробки головного компоненту системи, це моделі сайту (engine). Я прийняв рішення все робити за допомогою ООП (об'єктно орієнтоване програмування). І звичайно доречно було користуватися патероном MVC (model view controller), який досить широко використовується при розробці як невеличких так і крупних веб проектів, або проектів, де відбувається <<спілкування>> із клієнтом через інтерфейс (GUI).
\par Я прикинув собі структуру майбунього сайту на листочку і почав його розробку. Я вирішив робити <<user friendly>> посилання, тому перш за все створив файл .htaccess і вписав в нього правило mod\_rewrite.

\begin{lstlisting}[language=PHP]
RewriteEngine on

RewriteCond %{REQUEST_FILENAME} !-f
RewriteCond %{REQUEST_FILENAME} !-d

RewriteRule ^(.*)$ index.php?rt=$1 [L,QSA]
\end{lstlisting}

Звідcи добре видно, що задані правила, які дадуть нам можливість працювати із адресом типу http://example.com/news/view/id/1, а в істинності скріпт буде отримувати змінну \$rt. І вже її я розбираю в php, витягую звідти всі потрібні мені параметри. Перший параметр - це мій контролер, другий - це подія (action). В моєму прикладу це news та view відповідно. Далі я розбираю інші параметри, записуючи їх в асоціативний масив, і получаю до них доступ через ключ. Для прикладу я запрошую ключ id, мені повертається значення 1. Можна задавати довільну кількість параметрів.

\subsection{Програмна оболонка} 
Як відомо, php це не повністю об'єктно орієнтована мова програмування, навідміну від JAVA для прикладу. Тому доступ до класів відбувається через деяку точку входу. У моєму випадку це завжди буде файл index.php. 

\subsubsection{index.php} 
Даний файл служить тільки  для загрузки мого роутера (створює екземпляр класу router.class.php), та таких значень змінних як, абсолютний шлях до директорії із файлами, та повний шлях на веб сайт. Включає файл init.php для роботи із класами.




\section*{Висновок}
\addcontentsline{toc}{section}{Висновок}
Текст висновку

%%% Bibliography
\catcode`"\active\def"{\relax}
\bibliographystyle{gost780s}
\bibliography{bibliography}{}
\end{document}
