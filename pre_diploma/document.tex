%%% initial
\documentclass[ukrainian,14pt,utf8,
%pointsection, % це хз для чого
nocolumnsxix,
nocolumnxxxi,
nocolumnxxxii,
floatsection, %підпис плаваючих об'єктів згдіно номера розліду (таблиці, рисунки)
hpadding=5mm, %відстань до рамки зліва, справа
vpadding=5mm %відстань до рамки зліва, справа
]{eskdtext}

%%% for working with math
%\usepackage{amsmath}
%\usepackage{amssymb}

%%% for working some LaTeX packages
\usepackage{xecyr}

%%% for fonts
\usepackage{xltxtra}
%%% Times New Roman - main font
\setmainfont[Mapping=tex-text]{Times New Roman}
%%% Courier New - fot math
\setmonofont[Scale=MatchLowercase]{Courier New}

%%% for ---, --, << >> и т.п.
\defaultfontfeatures{Mapping=tex-text}

%%% ukrainian text
\usepackage{polyglossia}
\setdefaultlanguage{ukrainian}
\newfontfamily\ukrainianfont{Times New Roman}

%%% for words wrapping
%\XeTeXinterchartokenstate=1
%\XeTeXcharclass `\- 24
%\XeTeXinterchartoks 24 0 ={\hskip\z@skip}
%\XeTeXinterchartoks 0 24 ={\nobreak}

% для абзацу %
\usepackage{indentfirst}

% code highlighting %
\usepackage{listings}
\lstset{
language=Java,
basicstyle=\footnotesize\sffamily,
numbers=left,
numberstyle=\tiny,
frame=tb,
columns=fullflexible,
showstringspaces=false,
breaklines=true 
}

% automaticaly wrap to other line
\sloppy
%\usepackage{setspace}
%\onehalfspacing
\linespread{1.5}

%%% for graphix
\usepackage{graphicx}
\graphicspath{{images/}} %path to images
%%% titile for pictures
\addto{\captionsukrainian}{\renewcommand{\figurename}{Рисунок}}


%%% for internet ulrs
\usepackage{url}

%%% subtitiles(\subsubsection) don't show in article
%\setcounter{tocdepth}{2}

%%% new section from new page
\let\stdsection\section
\renewcommand\section{\newpage\stdsection}

%%% плаваючі обєкти підпис
%\renewcommand\section{\newpage\stdsection}

%%% Numering subtopics begin with B (B.1)
%\makeatletter
%\renewcommand\thesubsection{\ifnum\c@section=0{В.\arabic{subsection}}\else{\arabic{section}.\arabic{subsection}}\fi}
%\makeatother

%%% for coloring rows
\usepackage[table]{xcolor}

%%% для розриву таблиць 
\usepackage{longtable}
%\usepackage{eskdgraph}






\ESKDtitle{ Аналіз розробки програмного алгоритмічного забезпечення багатофункціональної корпоративної системи для сумісної роботи, управління документами і проектами }
\ESKDdocName{\uppercase{Звіт з переддипломної практики}}
\ESKDgroup{ІНФТУНГ ПЗ-07-1}

\ESKDauthor{ Бойчук Я.В. }
\ESKDchecker{ Бандура В.В. }
\ESKDsignature{ ПП.ПЗ 03.00.00.000 ПЗ }

%%finixg ukrainian languale%%
\renewcommand{\ESKDcolumnXfIname}{Розробив}
\renewcommand{\ESKDcolumnXfIIname}{Перевірив}
\renewcommand{\ESKDcolumnXfIVname}{Т.контр}
\renewcommand{\ESKDcolumnXfVname}{Н.контр}
\renewcommand{\ESKDcolumnXfVIname}{Затв.}

\renewcommand{\ESKDcolumnIVname}{Літ.}
\renewcommand{\ESKDcolumnVIIIname}{Аркушів}
\renewcommand{\ESKDcolumnXVIIname}{Підпис}
\renewcommand{\ESKDcolumnXVname}{Арк.}

%%fix ESKDTitle font
%%FIXME change for row, now for font
\renewcommand{\ESKDfontVsize}{%
\ESKDfontSetBaseLineStretch
\fontsize{10pt}{8pt}\selectfont\ESKDfontShape}

%%space after text
\setlength{\ESKDsectionSkipBefore}{-7mm \@plus -3mm \@minus -2mm}
\setlength{\ESKDsectionSkipAfter}{7mm \@plus 1mm \@minus 2mm}
\setlength{\ESKDsubsectionSkipBefore}{-7mm \@plus -3mm \@minus -2mm}
\setlength{\ESKDsubsectionSkipAfter}{7mm \@plus 1mm \@minus 2mm}
\setlength{\ESKDsubsubsectionSkipBefore}{-7mm \@plus -3mm \@minus -2mm}
\setlength{\ESKDsubsubsectionSkipAfter}{7mm \@plus 1mm \@minus 2mm}

%%font size for section (&&sub)
\renewcommand{\ESKDsectionStyle}{\centering\normalfont\bfseries\MakeUppercase}
\renewcommand{\ESKDsubsectionStyle}{\centering\normalfont\bfseries}
\renewcommand{\ESKDsubsubsectionStyle}{\centering\normalfont\bfseries}




        



\begin{document}
\fontsize{14}{14pt}\selectfont

\maketitle


\tableofcontents


%\section*{Вступ}
%\addcontentsline{toc}{section}{Вступ}
%Текст вступу
%

\section{Вступ}
В нас час важливою економічною точкою опори для будь-якої комерційної організації є наявність певних факторів, які визнають чітку позицію компанії на ринку. До всіх цих чинників можна віднести багато варіантів, зокрема:
\begin{enumerate}
\item капітал підприємства;
\item матеріальна база;
\item технічна база;
\item кваліфікований персонал;
\item місце підприємства на внутрішньому і зовнішньому ринку;
\item наявність сучасних засобів виробництва та ведення бізнесу;
\item та інші.
\end{enumerate}

Тому кожний розділ ведення бізнесу повинний бути детально розглянутий та впроваджений у життя. 
Але якщо подивитися із точки зору програмного забезпечення, то на даному етапі розвитку цивілізації, якісне ПЗ відіграє напевно найбільш важливу роль. 
Адже не можливо зараз утримувати всі дані в паперовому вигляді, не можливо відсилати друковані листи, чи спілкуватися тільки по телефоні і взнавати новини компанії тільки при зустрічі. 
У наш стрімкий час розвитку, новини міняються із колосальною швидкістю, тому встигнути за всім просто не можливо без певного програмного продукту. 
Уявіть собі інформатор, який сповіщає будь-які для Вас новини чи корисну інформацію в зручний для Вас час, при цьому вміє фільтрувати і аналізувати дані із попередніх запитів. 
Також на даний момент важко уявити не можливість спільної роботи над документами, над електронними таблицями. 
Дані технології вже давно використовуються людьми і підприємствами, починаючи від найменших де працює двоє людей, до величезних із кількістю працівників більше ста тисяч. 
Але для цього всього використовуються дуже багато технологій, які важко налаштувати і потребують великих витрат на підтримку.
Тому було розроблено багато сервісів і додатків, які полегшують роботу в мережі для підприємств.
\par Дане ПЗ використовується у всіх нішах нашого життя, починаючи від шкіл і лікарень, закінчуючи величезними корпораціями з будівництва космічних кораблів. 
Тому розробити універсальний продукт, який забезпечить всі вимоги, просто не можливо. 
Для кожної сфери існують свої нюанси.
\par Цікавою нішею для дослідження стало корпоративне програмне забезпечення для малого і середнього бізнесу.
 На даний момент існує багато програмних продуктів для комерційних цілей, проте вони здебільшого розраховані на великі корпорації і підприємства.
Тому використання їх для менших фірм просто не доцільно, або дуже складно із фінансової сторони (витрати на підтримку передують вигоді).
Як відомо, на ринку до цих пір зберігається тенденція на попит на корпоративне програмне забезпечення, яке б відповідало вимогам малого і середнього бізнесу, і в той же час було практично придатним для використання у великих корпоративних цілях

%TODO придумати назву заголовку. Може включити у підрозділ вступ
\section{Огляд програмного продукту}
Розглянемо більш детально пункт про наявність сучасних засобів ведення бізнесу. 
Кожна компанія, завжди стикається із проблемою ведення обліку працівників, ведення обліку фінансів, спільної роботи над документами та іншим.
Також є величезна і невід'ємна потреба у спільному доступі до документів, до корпоративного календаря, до блогу користувачів, до електронних таблиць та інформаційної дошки.
\par Портал підприємства (також відомий як enterprise information portal (EIP) або корпоративний портал) є основою для інтеграції інформації, людей і процесів в рамках організації. 
Це дає змогу забезпечити єдину точку доступу, часто у вигляді веб-інтерфейсу і призначеної для агрегування та персоналізації інформації за допомогою конкретних програмних додатків. Однією відмінною рисою корпоративних порталів є децентралізоване внесення конвенту та управління, яка зберігається на віддаленому сервері та постійно оновлюється.




\section*{Висновок}
\addcontentsline{toc}{section}{Висновок}
Текст висновку

%%% Bibliography
\catcode`"\active\def"{\relax}
\bibliographystyle{gost780s}
\bibliography{bibliography}{}
\end{document}
