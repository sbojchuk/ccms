%TODO
%extract every sectoin to files, 'cause it's hard to handle such big file
%add points in main page title
%change font size in Рисунок 1
%add count item to Рисунок 1, should be section and them count number
%chnage - for --- where needed
%check for goog wrapping
%check for borders and font size and 1.5 space (print document there and in office)
%chek printing exuality as is in PDF http://forum.ixbt.com/topic.cgi?id=23:27770
%add italic to subsubsection

%%% initial
\documentclass[ukrainian,14pt,utf8,
%pointsection, % це хз для чого
nocolumnsxix,
nocolumnxxxi,
nocolumnxxxii,
floatsection, %підпис плаваючих об'єктів згдіно номера розліду (таблиці, рисунки)
hpadding=5mm, %відстань до рамки зліва, справа
vpadding=5mm, %відстань до рамки зліва, справа
a4paper
]{eskdtext}

\special{papersize=210mm,297mm}

%%% for working with math
%\usepackage{amsmath}
%\usepackage{amssymb}

%%% for working some LaTeX packages
\usepackage{xecyr}

%%% for fonts
\usepackage{xltxtra}

%%% use for formula counting for every section
\numberwithin{equation}{section} 

%%% Times New Roman - main font
\setmainfont[Mapping=tex-text]{Times New Roman}
%%% Courier New - fot math
\setmonofont[Scale=MatchLowercase]{Courier New}

%%% for ---, --, << >> и т.п.
\defaultfontfeatures{Mapping=tex-text}

%%% ukrainian text
\usepackage{polyglossia}
\setdefaultlanguage{ukrainian}
\newfontfamily\ukrainianfont{Times New Roman}

%%% for words wrapping
%\XeTeXinterchartokenstate=1
%\XeTeXcharclass `\- 24
%\XeTeXinterchartoks 24 0 ={\hskip\z@skip}
%\XeTeXinterchartoks 0 24 ={\nobreak}

% для абзацу %
\usepackage{indentfirst}

% automaticaly wrap to other line
\sloppy
\usepackage{setspace}
%\onehalfspacing
\linespread{1.5}

%%% for graphix
\usepackage{graphicx}
\graphicspath{{images/}} %path to images


%%% for internet ulrs
\usepackage{url}

%%% subtitiles(\subsubsection) don't show in article
%\setcounter{tocdepth}{2}

%%% new section from new page
\let\stdsection\section
\renewcommand\section{\newpage\stdsection}

%%% плаваючі обєкти підпис
%\renewcommand\section{\newpage\stdsection}

%%% Numering subtopics begin with B (B.1)
%\makeatletter
%\renewcommand\thesubsection{\ifnum\c@section=0{В.\arabic{subsection}}\else{\arabic{section}.\arabic{subsection}}\fi}
%\makeatother

%%% for coloring rows
\usepackage[table]{xcolor}

%%% для розриву таблиць 
\usepackage{longtable}
\usepackage[]{caption} % table caption in center

\usepackage{listings}
\usepackage{color}
 
\lstset{ %
  basicstyle=\ttfamily\footnotesize,           % the size of the fonts that are used for the code 
  breaklines=true,                % sets automatic line breaking
  breakatwhitespace=false,        % sets if automatic breaks should only happen at whitespace
  keywordstyle=\bf{},          % keyword style
  commentstyle=\em{},       % comment style
}


\usepackage{enumitem} 
\setlist{nolistsep} % use for normal spacing for enumarate items, no vertical space

% \usepackage{tocloft}
% \renewcommand{\cftsecleader}{\cftdotfill{\cftdotsep}} % add dots to all table of contents

\usepackage[titles]{tocloft}
\renewcommand{\cftsecleader}{\cftdotfill{\cftdotsep}} % add dots to all table of contents

\usepackage{eskdtotal}

\usepackage{afterpage}

% \usepackage{scrpage2}
\usepackage{multirow}

\ESKDtitle{ фігня }
\ESKDdocName{ фінгня }

\ESKDauthor{ Бойчук Я.В. }
\ESKDchecker{ Бандура В.В. }
%\ESKDnormContr{  }


\ESKDdepartment{ s }
\ESKDcompany{ s }
\ESKDclassCode{ s }
\ESKDsignature{ ПЗ-07-1 }

\ESKDdate{ 2012/06/03}



\begin{document}
%і\fontsize{14}{14pt}\selectfont

%\maketitle


\tableofcontents


%\section*{Вступ}
%\addcontentsline{toc}{section}{Вступ}
%Текст вступу
%

\section*{ВСТУП}
\addcontentsline{toc}{section}{ВСТУП}
В нас час важливою економічною точкою опори для будь-якої комерційної організації є наявність певних факторів, які визнають чітку позицію компанії на ринку. До всіх цих чинників можна віднести багато варіантів, зокрема:
\begin{enumerate}
\item капітал підприємства;
\item матеріальна база;
\item технічна база;
\item кваліфікований персонал;
\item місце підприємства на внутрішньому і зовнішньому ринку;
\item наявність сучасних засобів виробництва та ведення бізнесу;
\item та інші.
\end{enumerate}

Тому кожний розділ ведення бізнесу повинний бути детально розглянутий та впроваджений у життя. 
Але якщо подивитися із точки зору програмного забезпечення, то на даному етапі розвитку цивілізації, якісне ПЗ відіграє напевно найбільш важливу роль. 
Адже не можливо зараз утримувати всі дані в паперовому вигляді, не можливо відсилати друковані листи, чи спілкуватися тільки по телефоні і взнавати новини компанії тільки при зустрічі. 
У наш стрімкий час розвитку, новини міняються із колосальною швидкістю, тому встигнути за всім просто не можливо без певного програмного продукту. 
Уявіть собі інформатор, який сповіщає будь-які для Вас новини чи корисну інформацію в зручний для Вас час, при цьому вміє фільтрувати і аналізувати дані із попередніх запитів. 
Також на даний момент важко уявити не можливість спільної роботи над документами, над електронними таблицями. 
Дані технології вже давно використовуються людьми і підприємствами, починаючи від найменших де працює двоє людей, до величезних із кількістю працівників більше ста тисяч. 
Але для цього всього використовуються дуже багато технологій, які важко налаштувати і потребують великих витрат на підтримку.
Тому було розроблено багато сервісів і додатків, які полегшують роботу в мережі для підприємств.
\par Дане ПЗ використовується у всіх нішах нашого життя, починаючи від шкіл і лікарень, закінчуючи величезними корпораціями з будівництва космічних кораблів. 
Тому розробити універсальний продукт, який забезпечить всі вимоги, просто не можливо. 
Для кожної сфери існують свої нюанси.
\par Цікавою нішею для дослідження стало корпоративне програмне забезпечення для малого і середнього бізнесу.
 На даний момент існує багато програмних продуктів для комерційних цілей, проте вони здебільшого розраховані на великі корпорації і підприємства.
Тому використання їх для менших фірм просто не доцільно, або дуже складно із фінансової сторони (витрати на підтримку передують вигоді).
Як відомо, на ринку до цих пір зберігається тенденція на попит на корпоративне програмне забезпечення, яке б відповідало вимогам малого і середнього бізнесу, і в той же час було практично придатним для використання у великих корпоративних цілях


\section{АНАЛІЗ РИНКУ І ПОТОЧНИХ РІШЕНЬ}
На даний момент існує досить багато готових рішень корпоративних порталів. 
Вони можуть забезпечувати підприємства всіма необхідними функціями і додатками, починаючи від системи обліку працівників, і завершуючи системою аналітики і збору даних, все це залежить від потреб ринку і певної компанії.
Проте майже всі вони здебільшого призначені для великих компаній, і невеличкі компанії або повинні витрачати величезні гроші на покупку ліцензій, або ж не користуватися усіма перевагами корпоративного порталу.
Тому було проведено загальний огляд продуктів і виділено основні переваги і недоліки, також виділено поточну використовувану ліценцію розповсюдження ПЗ.
\subsection{Переваги та недоліки поточних рішень}
Зробивши аналіз даної сфери, можна виділити декілька основних аспектів, які будуть використані для розробки подальшого програмного продукту.
Левова частка програмного забезпечення корпоративних порталів розроблено згідно стандартів\cite{portlet2}. 
Всі додатки і аплікації можуть без проблем взаємодіяти між собою. 
Проте великою їх нестачею для малої сфери бізнесу є закритість програмного коду і величезна вартість ліцензій.
Тому невеличкі компанії (до 100 людей) просто не мають змоги собі дозволити таку <<розкіш>>.

\subsubsection{Ліцензіювання і відкритість АРІ}
\par Було проведено аналіз поточних продуктів і їх ліцензій, і виявлено, що майже 90\% використовує проприєтарні рішення.
Більш детально розглянуто у таблиці \ref{t:portals}:

{\footnotesize
\begin{longtable}{|c|c|c|}
\captionsetup{justification=centering}
\caption{Список корпоративних порталів і використовувана ліцензія}\label{t:portals}\\
\hline
\multicolumn{1}{|c|}{\textbf{Назва продукту}}&
\multicolumn{1}{c|}{\textbf{Технологія}}&
\multicolumn{1}{c|}{\textbf{Ліцензія}}\\\hline

\endfirsthead
\caption*{\hfill Продовження таблиці \ref{t:portals}}\\\hline

\multicolumn{1}{|c|}{\textbf{Назва продукту}}&
\multicolumn{1}{c|}{\textbf{Технологія}}&
\multicolumn{1}{c|}{\textbf{Ліцензія}}\\\hline
\endhead

 Jetspeed & Java EE & Apache License v2.0\\ \hline
 ATG Portal & Java EE & Proprietary\\ \hline
 Backbase Portal Software & Java EE, .NET & Proprietary \\ \hline 
 Broadvision Portal & Java EE & Proprietary \\ \hline 
 Bluenog ICE & Java EE & Proprietary \\ \hline
 enPortal  & Java EE & Proprietary\\ \hline 
 CommunityManager.NET  & .NET & Proprietary \\ \hline
 eXo Portal & Java EE & Affero General Public License \\ \hline
 eXo Platform & Java EE & Proprietary \\ \hline
 GateIn Portal & Java EE & LGPL \\ \hline
 Hippo CMS & Java EE & Open Source and Proprietary Licenses \\ \hline
 WebSphere Portal & Java EE & Proprietary \\ \hline
 TeamPortal & Java EE & Proprietary \\ \hline
 JBoss Enterprise Portal  & Java EE & LGPL \\ \hline
 IntraNet & ASP.NET & Proprietary \\ \hline
 Liferay Portal & Java EE & Proprietary Licenses \\ \hline
 TeamWox Groupware & C++ & Proprietary \\ \hline
 SharePoint Server & ASP.NET & Proprietary \\ \hline
 Vignette Portal 8.0 & Java EE & Proprietary \\ \hline
 Oracle WebCenter Suite 11g & Java EE & Proprietary \\ \hline
 Oracle WebLogic Portal 10g & Java EE & Proprietary \\ \hline
 Oracle WebCenter Interaction 10g & ASP.NET & Proprietary \\ \hline
 Oracle IAS Portal 10g & Java EE & Proprietary \\ \hline
 Regroup & Ruby & Proprietary \\ \hline
 ACUBE Portal 5.0 & Java EE & Proprietary \\ \hline
 SAP NetWeaver 7.0 & Java EE & Proprietary \\ \hline
 SORCE V9 & ASP.NET & Proprietary \\ \hline
 Sun Java System Portal Server & Java EE & Open Source, licensing \& support plans \\ \hline
 Sun GlassFish Web Space Server  & Java EE & Open Source, licensing \& support plans \\ \hline
 tmsEKP 1.52 & Java EE & Proprietary \\ \hline
 PortalBuilder 5.2 & Java EE & Proprietary \\ \hline
 ProPortal 4.0 & Java EE & Proprietary \\ \hline
 Intrexx & Java EE & Proprietary \\ \hline
 uPortal & Java EE & Apache License v2.0 \\ \hline

\end{longtable}
}


Зробивши певний аналіз, можна дійти до висновку, якби невеликим компаніям давати можливість використовувати продукт на підставах вільного розповсюдження коду, то вони б охотніше пробували, і з часом інвестували в нього гроші, або просто <<купляти>> підтримку.
Тобто використати одну із відкритих ліцензій типу GNU Public License.
\par Економічна вигода продукту буде базуватися на корпоративній платній підтримці, типу все починаючи від підбору серверів -- до налаштування і підтримки продукту.
Зате розробники зможуть у загальний репозиторій додавати свої зміни та виправляти помилки, що значною мірою пришвидшить процес розробки.
Основна стратегія розробки буде націлена на швидкий вихід на ринок і пошук потенційних клієнтів.
Також велика увага буде прикута для ринку пост радянських республік, адже на даний момент ринок бізнесу стрімко розвивається, тобто попит є, а пропозиція не повній мірі відповідає потребі.
Портали які розробляються переважно націлені на Європейський та Американській ринок, а також Азію.
Тому базуючись на цьому було виділено такі основні вимоги, як врахування нашого законодавства (для прикладу по працевлаштуванню працівників, ведення документації, конвертації валют і тому подібне) та локалізацію сервісу.
Тим більше підтримка користувачів буде набагато легше і ефективніше всередині країни, ніж з-за кордону, що дасть нам перевагу над іншими існуючими продуктами.

\par Також велику увагу буде приділено відкритості АРІ для взаємодії із вже існуючими додатками.
Адже існуючі рішення в основному базуються на закритих протоколах, чим саме змушують користувачів прив'язуватися до їхньої системи і залежати від них
\par Інтерфейс і система всіх сучасних продуктів дуже <<важка>> і мульти-функціональна, що потребує значних ресурсів як у користувачів так і на стороні сервера.
Цей аспект також буде максимально спрощений, що в свою чергу дозволить виділитися продукту на ринку малого бізнесу.

\subsection{Технології розробки} 
Базуючись на поточних стандартах \cite{portlet2} та використовуваних базових технологій (таблиця \ref{t:portals}) було прийнято рішення впровадити розробку на базі Java.
Адже саме Sun (зараз Oracle) <<диктує>> моду на ринку стандартизації портлетів, тому буде просто пристосуватися до поточних рішень.
\par Звичайно ж буде використано всі переваги Java EE.
Для гнучкої і швидкої розробки буде застосовано Spring Framework із ORM обгорткою Hibernate поверх бази даних MySQL.
Для фронтенд логіки UI в основному буде використовуватися jQuery фреймворк.
Пошук забезпечить Apache Lucene.
Сервер бекенду буде працювати на Apache Tomcat.
\par На даний момент не планується стандартизація щодо портлетів, просто в майбутньому можливо буде виділено цей пункт для реалізації в системі.


\subsection{Технічний огляд продукту}
Розглянемо більш детально пункт про наявність сучасних засобів ведення бізнесу. 
Кожна компанія, завжди стикається із проблемою ведення обліку працівників, ведення обліку фінансів, спільної роботи над документами та іншим.
Також є величезна і невід'ємна потреба у спільному доступі до документів, до корпоративного календаря, до блогу користувачів, до електронних таблиць та інформаційної дошки.
\par Портал підприємства (також відомий як enterprise information portal (EIP) або корпоративний портал) є основою для інтеграції інформації, людей і процесів в рамках організації. 
Це дає змогу забезпечити єдину точку доступу, часто у вигляді веб-інтерфейсу і призначеної для агрегування та персоналізації інформації за допомогою конкретних програмних додатків. Однією відмінною рисою корпоративних порталів є децентралізоване внесення контенту та управління, яка зберігається на віддаленому сервері та постійно оновлюється.
%TODO змінити якось =)
\subsection{Історичний огляд корпоративної сфери}
В середині 1990-х років появилися громадські такі веб-портали як AltaVista, AOL, Excite і Yahoo!. 
Вони забезпечували користувачів певним набором функцій (наприклад новини, електронна пошта, погода, котирування акцій і пошук), які часто були представлені у вигляді автономного порталу.
Незабаром підприємства усіх типів і форм почали бачити необхідність аналогічного функціоналу для їх різноманітних потреб, проте із єдиною точкою доступу.
\par До кінця 1990-х років, виробники програмного забезпечення почали розробляти веб-портали для різних підприємств. 
Ці програмні пакети були розроблені таким чином, щоб підприємства могли легко розгортати свої власні налаштування корпоративного порталу та доповнювати його своїми додатками.
Перші постачальники комерційних веб порталів з'явилися в 1998 році, це були такі фірми як: Epicentric, Plumtree  та Viador. 
Ці фірми були основними гравцями на ринку, проте ситуація змінилася в 2002 року, коли на ринок почали виходити постачальники серверних аплікацій, такі як BEA, IBM, Passageways, Oracle Corporation and Sun Microsystems.
Підприємства можуть вибрати для своїх цілей декілька порталів, що базується на основі їх бізнес-структури та стратегічної спрямованості.
\par У 2003 році розробники Java-порталів випустили стандарт, відомий як JSR-168. 
Він повинен був визначити API для взаємодії між корпоративних порталів та портлетів.
Постачальники програмного забезпечення почали розробляти JSR-168 сумісні портлети, які можуть бути розгорнуті на будь-якому JSR-168 сумісному корпоративному порталі. 
Другий ітераційний стандарт JSR-286 є остаточним на даний момент і випущений 12 червня 2008 року.

\subsection{Портлети}
Портлет -- це змінний компонент інтерфейсу веб-порталу (елемент веб--сторінки), який можливо певним чином підключити до порталу.
Портлет містить в собі фрагменти розмітки, які вбудовуються в сторінку порталу. 
Найчастіше сторінка порталу представляється у вигляді набору портлетів, які взаємодіють між собою. 
Таким чином, портлет (або сукупність портлетів) представляється у вигляді єдиного веб-додатку, розміщеного на порталі. 
Приклади портлетів можуть бути наступними: електронна пошта, повідомлення про погоду, фінансовий стан, останні новини і тому подібне.
Завдяки існуючим стандартам розробники можуть створювати портлети, що легко вбудовуються в будь-який портал, який слідує стандартам і правилам.
\par Існує протокол WSRP, що забезпечує стандарт веб-сервісів, що дозволяє автоматично вбудовувати віддалено запущені портлети з різних джерел.
Специфікація Java-портлетів JSR168 дає можливість взаємодіяти між собою портлети з різних веб-порталів. 
Ця специфікація визначає безліч API для взаємодії контейнерів портлетів і дає різні адреси областей персоналізації, подання та безпеки.
Існує безліч постачальників комерційних контейнерів портлетів. 
Як відомо лідирують у цій галузі IBM, Oracle, Vignette. 
Реалізації від цих постачальників мають додаткові розширення і налаштування, проте деякі із ним можуть бути не затверджені стандартами. 
Крім того, є портали з відкритим вихідним кодом, що підтримують JSR168, такі як корпоративний портал Apache Jetspeed-2 або eXo Portal.
\subsubsection{Apache Pluto}
Розглянемо на прикладі один з найбільш вдалої реалізації стандарту портлетів JSR168 -- це Apache Pluto.
Портлет працює всередині контейнера портлетів (Pluto).
Цей контейнер містить портлет з необхідним середовищем для подальшого виконання.
Контейнер портлетів керує життєвим циклом всіх вікон порталу та надає інтерфейси для портлетів, котрі викликаються всередині нього.
Контейнер також запускає методи на виконання із доступних цільових користувацьких сторінок, і взаємодіє із сторінками порталу. Принцип роботи і архітектурні компоненти аплікації Pluto 2.0. зображено на рисунку \ref{pic:pluto}.

    \begin{figure}[!ht]
		\centering
		\includegraphics[width=1.00\textwidth]{pluto.png}
		\captionof{figure}{Принцип роботи Pluto}\label{pic:pluto}
	\end{figure}


\par В даному випадку, Pluto вбудований безпосередньо в корпоративний портал. 
Потім через перехресний запит (через веб-додатки) відбувається відправлення запиту для відображення вмісту портлету, який як правило знаходяться в різних додатках на порталі і в контейнерах. 


%TODO хз чи тут потрібно загальний розділ, поки забрав
\subsection{Система керування вмістом}
Система управління контентом (content management system - CMS) дозволяє публікувати, редагувати і змінювати вміст веб-сторінок, а також обслуговувати портал з центральної сторінки. 
При цьому надається набір процедур, що використовуються для управління робочим процесом у середовищі для спільної роботи.
Вони можуть бути ручні або комп'ютеризовані (в автоматичному режимі).
\subsubsection{Головні функції CMS}
До основних функцій можна віднести наступні пункти:
\begin{itemize}
\item можливість великій кількості людей ділитися інформацією і робити свій вклад в розвиток порталу;
\item контроль доступу до даних на основі ролей користувачів (наприклад визначити роль, яка має тільки права на перегляд інформації, або ж редагування, публікацію тощо);
\item пошук і поширення інформації між користувачами;
\item зменшення дублікацій на вході;
\item спрощене керування корпоративними додатками;
\item відносно легка комунікація між користувачами.
\end{itemize}

\subsubsection{Типи даних та їх використанням}
У CMS дані можуть бути представлені як правило у будь-якій формі: документи, відео, тексти, фотографії, номери телефонів, наукові дані і тому подібне. 
CMS часто використовуються для зберігання, управління, перегляду і публікації документів. 
Також досить поширене використання в якості центрального сховища у зв'язці із централізованою системою контролю версій, що є однією із переваг CMS.
\subsubsection{Управління корпоративною інформацією}
Enterprise Content Management (ECM) - управління інформаційними ресурсами підприємства або управління корпоративною інформацією.
В даному контексті інформація (контент) передбачається як слабо структурована одиниця - це можуть бути файли різних форматів, електронні документи з різними наборами полів і т. п.
За визначенням ECM - це стратегічна інфраструктура і технічна архітектура для підтримки єдиного життєвого циклу неструктурованої інформації різних типів і форматів. 
ECM-системи складаються з додатків, які можуть взаємодіяти між собою, а також використовуватися і продаватися самостійно. 
\par Всі сучасні ECM-системи визначають такі ключові компоненти:
\begin{itemize}
\item управління документами --- довгострокове архівування, автоматизація політик зберігання та відповідності нормам регулюючих органів, забезпечення відповідності законодавчим та галузевим нормам;
\item управління веб-контентом (WCM) --- автоматизація ролі веб-майстра, управління динамічним контентом і взаємодією між користувачами;
\item  управління мультімедіаконтентом (DAM) --- управління графічними, відео та аудіофайлами, різними маркетинговими матеріалами, наприклад, флеш-банерами, рекламними роликами;
\item управління знаннями (Knowledge Management) --- підтримка систем для накопичення та доставки релевантної для бізнесу інформації;
\item документо-орієнтоване взаємодія (співробітництво) --- спільне використання документів користувачами та підтримка проектних команд.
\end{itemize}

%TODO якщо буде мало тексту, вставити http://en.wikipedia.org/wiki/Enterprise\_content\_management
%\subsubsection{Компоненти ECM} 

\subsection{Система управління документами}
Система управління документами (DMS - Document management system) - комп'ютерна система (або набір комп'ютерних програм), що використовується для відстеження та зберігання електронних документів і / або образів (зображень та інших артефактів) паперових документів.
Дане поняття тісно пов'язане з концепцією Content Management System (система керування вмістом) і зазвичай розглядається як компонент Enterprise Content Management System (CMS рівня підприємства).
У загальному випадку системи управління документами (DMS) надають можливість зберігання, ведення контролю версій, позначення метаданими і безпеку по відношенню до документів, а також індексування і розвинені можливості пошуку документів.
\subsubsection{Метадані}
Метадані зазвичай зберігаються для кожного документа. 
Метадані, наприклад, можуть включати дату занесення документа в сховище і код користувача, котрий виконав зміни до файлу. 
Система управління документами також може витягувати метадані з документа автоматично або запитувати їх у користувача. 
Деякі системи надають сервіс оптичного розпізнавання тексту відсканованих документів, або можливість витягувати текст з електронних документів. 
Використовуючи опрацьований текст система дозволяє здійснювати пошук документа за ключовими словами всередині самого документа.

\subsubsection{Інтеграція }
Багато систем управління документами намагаються інтегрувати функцію управління документами безпосередньо в різні додатки, дозволяючи користувачеві отримувати документ відразу зі сховища системи управління документами, робити які-небудь модифікації, і зберігати його назад в сховище в якості нової версії, і все це проробляти в одному додатку, не виходячи з нього. 
Дана інтеграція в основному доступна для офісних пакетів і поштових клієнтів або для програмного забезпечення, призначеного для групової або колективної роботи. 
Інтеграція зазвичай має на увазі використання таких відкритих стандартів як: ODMA, LDAP, WebDAV і SOAP.

\subsubsection{Захоплення тексту}
Під захопленням тексту мається на увазі переведення паперових документів в цифровий варіант за сканерів та МФУ.
Також часто використовується програмне забезпечення для оптичного розпізнавання тексту, щоб конвертувати цифрові зображення в текст.

\subsubsection{Індексування}
Індексування надає можливість класифікувати документи за допомогою метаданих і індексування словникового тексту, який було витягнутого з документа.
Індексація існує для підтримки розвинених можливостей пошуку документів. 
Одна з головних умов швидкого та якісного пошуку - це створення індексу документа.

\subsubsection{Сховище}
Основне призначення це для зберігання електронних версіях документів. 
Сховище документів також включає в себе і керування тими ж документами, котрі воно зберігає.
Також сховище забезпечує міграцію з одного носія на інший і забезпечує цілісність даних.
Сховище документів може бути як файлове, так і сховище у вигляді СУБД (бази даних). 
У свою чергу, сховище документів в СУБД може бути як в одній базі даних, так і в окремо розподілених базах даних.
        

\subsection{Програмне забезпечення для спільної роботи}
Програмне забезпечення для спільної роботи (англ. collaborative software, groupware, workgroup support systems, group support systems) - програмне забезпечення створене з метою підтримки взаємодії між людьми, котрі спільно працюють над вирішенням деяких спільних завдань. 

\subsubsection{Огляд ПЗ для спільної роботи}
Програмне забезпечення для спільної роботи -- це область, яка в значній мірі перекривається з областю CSCW (англ. computer-supported cooperative work (CSCW)).
Часто вважається що ці області еквівалентні, хотя з іншого боку програмне забезпечення для спільної роботи є підчастиною CSCW.
Сюди відносяться такі системи як: електронна пошта, календарі, текстовий чат, вікі сторінки, корпоративні закладки, блог.
Оскільки ПО спільної роботи відноситься до технологічних елементів CSCW, системи спільної роботи стають корисним аналітичним інструментом у вивченні поведінкових і організаційних параметрів, пов'язаних з більш широкою сферою CSCW.

%FIXME забрати перекладений текст, хтось колись в неті перекладав
\subsubsection{Види взаємодії}
В літературі можна зустріти кілька різних визначень спільної роботи (англ. - collaboration) в застосуванні до інформаційних технологій. Деякі з них виправдані, інші ж настільки великі, що починають втрачати будь-який сенс.
Для того щоб бути впевненим що обрані технології підходять для конкретних потреб, необхідно розуміти відмінності в способах взаємодії людей один з одним.
Є три основні шляхи, по яких здійснюється взаємодія між людьми: 
\begin{itemize}
\item діалог;
\item здійснення угоди;
\item співробітництво.
\end{itemize}

\par Діалог - це обмін інформацією між одним або кількома учасниками, основна мета якого полягає у з'ясуванні їх позицій і встановлення взаємин. 
Відбувається вільний обмін інформацією без будь-яких обмежень. 
Для підтримання діалогу цілком підходять звичайні комунікаційні технології, такі як телефон, миттєві повідомлення та електронна пошта.
\par Укладення угоди передбачає обмін якимись сутностями, і ця процедура зазвичай проводиться за добре певними правилами і передбачає зміну відносин між учасниками. Наприклад, один з учасників угоди обмінює гроші на товари і стає покупцем. Новий статус учасників операції та обмінюваних сутностей потрібно зберегти в будь-якому надійному сховищі. Такі операції добре обслуговуються системами управління транзакціями. 
\par Співпраця полягає в тому, що його учасники обмінюються якимись загальними сутностями (на противагу угоді, коли предмет обміну належить лише одному учаснику). 
Як приклад можна привести просування нової ідеї, створення нової конструкції, досягнення спільних цілей. 
При цьому самі сутності досить розпливчасті і невизначені. 
Таким чином, технології для забезпечення спільної роботи теж повинні бути достатньо гнучкими. 
Вони повинні включати в себе управління документами, кошти для ведення обговорень з можливістю сортування за темами, можливість відновити історію внесених змін та багато іншого.
 
\subsubsection{Рівні взаємодії}

Рівні взаємодії можна поділити на три категорії по рівню забезпечення взаємодії: засоби зв'язку, засоби для організації конференцій та засоби управління.

\par Електронні засоби зв'язку використовуються для пересилання повідомлень, файлів, даних чи документів між людьми і таким чином дають можливість для обміну інформацією:
\begin{itemize}
\item електронна пошта;
\item факс;
\item голосова пошта;
\item веб-публікації.
\end{itemize}

Електронні конференції також дають змогу для обміну інформацією, проте в інтерактивній формі це є:
\begin{itemize}
\item телефонні конференції;
\item відео і аудіо конференції;
\item інтернет форуми;
\item чати.
\end{itemize}

Засоби управління діяльність групи:
\begin{itemize}
\item електронні календарі (створення щоденників, системи автоматичного нагадування);
\item системи управління проектами (складання розкладу робіт, відслідковування їх виконання);
\item управління документообігом;
\item бази знань - збір, сортування, зберігання і організація доступу до різних форм інформації.
\end{itemize}





\subsection{Інтранет}
Інтранет (англ. Intranet, також вживається термін інтрамережа) - на відміну від мережі Інтернет, це внутрішня приватна мережа організації. 
Як правило, Інтранет - це Інтернет в <<мініатюрі>>, який побудований на використанні протоколу IP для обміну і спільного використання деякої частини інформації всередині певної організації. 
Це можуть бути списки співробітників, списки телефонів партнерів і замовників. 
Найчастіше під цим терміном мають на увазі тільки видиму частину Інтранет - внутрішній веб-сайт організації. 
Заснований на базових протоколах HTTP і HTTPS і організований за принципом клієнт-сервер, інтранет-сайт доступний з будь-якого комп'ютера через браузер. 
\par Таким чином, Інтранет - це <<приватний>> Інтернет, обмежений віртуальним простором окремо взятої організації. 
Intranet допускає використання публічних каналів зв'язку, що входять в Інтернет, (VPN), але при цьому забезпечується захист переданих даних і мають набір заходів щодо припинення проникнення ззовні на корпоративні вузли.
\par Програми в Intranet засновані на застосуванні Інтернет-технологій і особливо Web-технології: гіпертекст у форматі HTML, протокол передачі гіпертексту HTTP і інтерфейс серверних додатків CGI. 
Складовими частинами Intranet є Web-сервери для статичної або динамічної публікації інформації і браузери для перегляду й інтерпретації гіпертексту.

\subsubsection{Особливості, переваги та недоліки Інтранет}
Інтранет побудований на базі тих же понять і технологій, які використовуються для Інтернету, такі як архітектура клієнт-сервер і стек протоколів Інтернету (TCP / IP). 
В Інтранет зустрічаються все з відомих інтернет-протоколів, наприклад, протоколи HTTP (веб-служби), SMTP (електронна пошта) і FTP (передача файлів). 
Інтернет-технології часто використовуються для забезпечення сучасними інтерфейсами функції інформаційних систем, які розміщують корпоративні дані.
\par Інтранет можна представити як приватну версію Інтернету, або як приватнe розширення Інтернету, обмеженого організацією за допомогою брандмауера. 
\par Перші інтранет-веб-сайти і домашні сторінки почали з'являтися в організаціях у 1990-1991 роках. 
Проте за неофіційними даними, термін Інтранет вперше почав використовуватися в 1992 році в таких закладах, як університети і корпорації, що працюють у технічній сфері.
\par Інтранет також протиставляють Екстранет, доступ до Інтранету надано тільки службовцям організації, в той час як до Екстранет можуть отримати доступ клієнти, постачальники, або інші затверджені керівництвом особи. 
В Екстранет-технології крім приватної мережі, користувачі мають доступ до Інтернет ресурсів, але при цьому здійснюються спеціальні заходи для безпечного доступу, авторизації, і аутентифікації.
\par Інтранет компанії не обов'язково повинен забезпечувати доступ до Інтернету. 
Коли такий доступ все ж забезпечується, зазвичай це відбувається через мережевий шлюз з брандмауером, захищаючи Інтранет від несанкціонованого зовнішнього доступу. 
Мережевий шлюз часто також здійснює аутентифікацію користувачів, шифрування даних, і часто - можливість з'єднання по віртуальній приватній мережі (VPN) що знаходяться за межами підприємства.

Переваги використання Інтранет:
\begin{itemize}
\item висока продуктивність при спільній роботі над якимись загальними проектами;
\item легкий доступ персоналу до даних;
\item гнучкий рівень взаємодії: можна міняти бізнес-схеми взаємодії як по вертикалі, так і по горизонталі;
\item миттєва публікація даних на ресурсах Інтранет дозволяє специфічні корпоративні знання завжди підтримувати у формі і легко отримувати звідусіль в компанії, використовуючи технології Мережі та гіпермедіа;
\item дозволяє проводити в життя загальну корпоративну культуру і використовувати гнучкість і універсальність сучасних інформаційних технологій для управління корпоративними роботами.
\end{itemize}


Переваги веб-сайту в Інтранет перед клієнтськими програмами архітектури клієнт-сервер:
\begin{itemize}
\item Не потрібно інсталяція програми-клієнта на комп'ютерах користувачів (як неї використовується браузер).
\item Відповідно, при змінах функціональності корпоративної інформаційної системи оновлення клієнтського ПЗ також не потрібно.
\item  Скорочення тимчасових витрат на рутинних операціях по вводу різних даних, завдяки використанню веб-форм замість обміну даними по електронній пошті
\item Крос-платформна сумісність.% - стандартний браузер на Microsoft Windows, Mac і GNU / Linux / * NIX.
\end{itemize}


Основні недоліки Інтранет:
\begin{itemize}
\item мережа може бути зламана і використана в хакерських цілях цілях;
\item неперевірена або неточна інформація, опублікована в Інтранет, призводить до плутанини і непорозумінь;
\item легкий доступ до корпоративних даних може спровокувати їх витік до конкурентів через несумлінного працівника;
\item працездатність і гнучкість Інтранет вимагають значних накладних витрат на розробку і адміністрування.
\end{itemize}








\subsection{Корпоративна Wiki}

Корпоративна вікі --- це програмне забезпечення яке призначене для використання в корпоративній сфері і служить особливим чином для підвищення внутрішнього обміну знаннями, з великим акцентом на такі функції, як контроль доступу, інтеграція з іншими програмними продуктами та управління документами. 
\par В організаціях вікі може або додати або замінити централізовану систему керування контентом. 
Її децентралізований характер дозволяє швидкому поширенню необхідної інформації в межах організації.
Вікі являється швидшим організаційним продуктом ніж централізований репозиторій знань.
Вікі може використовуватися для управління проектами, взаємодією з клієнтами, планування ресурсів підприємства а також інші види управління даними.

\par Особливості вікі для корпорації включають в себе такі основні аспекти як:
\begin{itemize}
\item швидкий і простий доступ для створення сторінок, які містять посилання на інші корпоративні системи;
\item дозволяє розвантажити електронну пошту за рахунок зберігання всієї необхідної інформації із можливістю спільного доступу людьми які є на даному проекті.
\item гнучка організація інформації;
\item швидкий і розширений пошук.
\end{itemize}



\subsection{Онлайн офіс}
Онлайн офіс --- це набір веб-сервісів у формі програмного забезпечення яке подану кінцевому користувачеві як послуга. 
Набір наданих веб-служб зазвичай включає всі основні можливості традиційних офісних пакетів, такі як текстовий редактор, електронні таблиці, додаток для створення презентацій, органайзер справ і навіть аналоги СУБД. 
Онлайн офіс може бути доступний з будь-якого комп'ютера, у якого є доступ в Інтернет, незалежно від того, яку операційну систему користувач використовує. 
Це дозволяє людям працювати разом по всьому світу і в будь-який час, що веде до створення міжнародних віртуальних команд для спільної роботи над проектами. 


\subsection{Корпоративний блог}
Корпоративний блог --- це блог, що видається організацією і використовується як для зв'язків з громадськістю, так і для внутрішньої організації. 
Або повністю підконтрольний організації, координований і наповнюється нею контентом, але формально з нею не пов'язаний.



\subsubsection{Внутрішньокорпоративний блог}
Внутрішній корпоративний блог --- це важливий засіб комунікації, особливо у великих компаніях. 
Можна навести деякі явні переваги:
\begin{itemize}
\item блог допомагає поліпшити взаємодію співробітників, надає можливості для навчання. Він добре підходить для запуску нових проектів, для роботи в неоднорідних, великих колективах;
\item блог допомагає виявити різні погляди на будь-яке питання. Відкритість для публікації постів і коментарів --- хороша можливість висловитися всім членам колективу;
\item шляхом дискусій на задану тему блог допомагає знайти компроміс при наявності різних точок зору.
Для керівників блог --- можливість налагодити взаємодію з співробітниками;
\item блог --- це своєрідна <<історія фірми>>, архів ідей і обговорень.
\item найчастіше кожен співробітник може залишити коментар до будь-якого посту. Коло авторів блогу визначається політикою компанії, часто написати пост може будь-який співробітник.
\end{itemize}


Блог має певні переваги перед такими внутрішньокорпоративними комунікаціями, як, наприклад, листування по електронній пошті, зокрема:

\begin{itemize}
\item коли листів стає занадто багато, це ускладнює спілкування;
\item не всі співробітники вміють правильно архівувати листи, в результаті чого вони не зможуть згодом знайти необхідну інформацію.
\end{itemize}

Внутрішній блог --- альтернатива чи доповнення до корпоративних зборів, нарад. 
Співробітники великих компаній часто не мають можливість проводити наради (наприклад, через велику відстань між філіями або зайнятості).

\subsubsection{Публічний блог}
Одна з основних цілей компаній --- це налагодження комунікацій з клієнтами (як поточними, так і потенційними).
Завдяки оперативності публікації постів і можливості коментування публічний корпоративний блог дуже важливий для досягнення цієї мети.
Блоги є цінним доповненням до корпоративного сайту, так як в них може бути представлена альтернативна точка зору на те чи інше питання, ті чи інші продукти компанії можуть бути описані більш простою і доступною мовою.

\section{АЛГОРИТМІЧНЕ ЗАБЕЗПЕЧЕННЯ ПРОЦЕСУ СТВОРЕННЯ КОРПОРАТИВНОЇ СИСТЕМИ}

TODO
Короче щось тут сказати про проектування, а то бошка щось не варить)))
В залежності від того, як буде спроектовано систему в цілому, буде

\subsection{Структура проекту}

\subsubsection{Проектування бази даних}
Як і кожний великий проект, даний проект повинний працювати із базою даних.

\begin{center}
		\includegraphics[width=1.00\textwidth]{db_schema.png}
		\captionof{figure}{Загальна схема структури бази даних}
\end{center}


\begin{center}
		\includegraphics[width=1.00\textwidth]{mockup_mainpage.png}
		\captionof{figure}{Макет майбутнього сайту}
\end{center}
\section{ПРОГРАМНА РЕАЛІЗАЦІЯ СИСТЕМИ ДЛЯ УПРАВЛІННЯ КОРИСТУВАЧАМИ, ДОКУМЕНТАМИ, ЗАВДАННЯМИ І МОЖЛИВОСТІ СПІЛЬНОЇ РОБОТИ}
\subsection{Реалізація роботи бази даних}
\par Загальна структура бази даних зображено на рисунку \ref{pic:db_shema}. Весь код для створення бази даних реалізовано мовою SQL, за допомогою запитів до БД.
\par Зовнішні посилання створено за допомогою команди Reference. Для прикладу код для створення таблиці користувачів (worker), котра має чотири зовнішні ключі які посилаються на таблиці команди (team), роль користувача (role\_name), тип посади (job\_type\_name) та регіон (region\_name).

\par Відповідно таким чином і реалізовано всі решта таблиць. Код для створення таблиці таблиці користувачів:
\begin{lstlisting}[language=SQL]
DROP TABLE IF EXISTS `worker`;
/*!40101 SET @saved_cs_client     = @@character_set_client */;
/*!40101 SET character_set_client = utf8 */;
CREATE TABLE `worker` (
  `id` bigint(20) NOT NULL AUTO_INCREMENT,
  `birthday` date DEFAULT NULL,
  `date_hire` date DEFAULT NULL,
  `login` varchar(255) NOT NULL,
  `name` varchar(255) NOT NULL,
  `pass` varchar(255) NOT NULL,
  `phone` varchar(255) DEFAULT NULL,
  `photo` longblob,
  `private_mail` varchar(255) DEFAULT NULL,
  `street` varchar(255) DEFAULT NULL,
  `surname` varchar(255) NOT NULL,
  `version` int(11) DEFAULT NULL,
  `job_type_name` bigint(20) NOT NULL,
  `region_name` bigint(20) NOT NULL,
  `role_name` bigint(20) DEFAULT NULL,
  `team_name` bigint(20) NOT NULL,
  `mobile` varchar(255) DEFAULT NULL,
  PRIMARY KEY (`id`),
  UNIQUE KEY `login` (`login`),
  KEY `FKD162537E30271785` (`team_name`),
  KEY `FKD162537EB0DD720C` (`job_type_name`),
  KEY `FKD162537E520C2F83` (`role_name`),
  KEY `FKD162537EBDECCB25` (`region_name`),
  CONSTRAINT `FKD162537E30271785` FOREIGN KEY (`team_name`) REFERENCES `team` (`id`),
  CONSTRAINT `FKD162537E520C2F83` FOREIGN KEY (`role_name`) REFERENCES `worker_role` (`id`),
  CONSTRAINT `FKD162537EB0DD720C` FOREIGN KEY (`job_type_name`) REFERENCES `worker_job_type` (`id`),
  CONSTRAINT `FKD162537EBDECCB25` FOREIGN KEY (`region_name`) REFERENCES `region` (`id`)
) ENGINE=InnoDB AUTO_INCREMENT=3 DEFAULT CHARSET=utf8;
/*!40101 SET character_set_client = @saved_cs_client */;	
\end{lstlisting}
\par Посилання на інші таблиці на веб інтерфейсі реалізовано за допомогою випадаючих списків -- це дає можливість забезпечити введення вірних даних і допомагає відобразити вже існуючі в базі даних записи.
Для прикладу візьмемо форму для створення нового користувача та вибору регіону, що зображено на рисунку \ref{pic:page_drop_down}.


\begin{figure}[!ht]
\centering
    \includegraphics[width=0.5\textwidth]{page_drop_down.png}
    \captionof{figure}{Вибір регіону при створенні користувача}\label{pic:page_drop_down}
\end{figure}


\subsection{Реалізація веб інтерфейсу}
\par Веб інтерфейс користувача повинний бути зручний та інтуїтивно зрозумілий кожному користувачеві, тому його було реалізовано в легких тонах та зручно розташовано всі навігаційні елементи.

\begin{figure}[!ht]
\centering
		\includegraphics[width=1.00\textwidth]{page_main.png}
		\captionof{figure}{Загальний інтерфейс програмного продукту}\label{pic:page_main}
\end{figure}
\par Весь інтерфейс сайту поділяється на три основні компоненти: головний блок, навігаційна панель, головне меню. Для зручності розробки цих компонентів використано Apache Tiles, що дає змогу поділяти код на логічні одиниці.
\begin{lstlisting}
<body>
 <div id="wrapper">
  <header>
   <tiles:insertAttribute name="header" ignore="true" />
  </header>
  <section>
   <div class="container_8 clearfix">
    <tiles:insertAttribute name="menu" ignore="true" />
    <tiles:insertAttribute name="body" />
   </div>
   <div id="push"><!-- --></div>
  </section>
 </div>
  <footer>
   <tiles:insertAttribute name="footer" ignore="true" />
  </footer>
 <div class="apple_overlay black" id="overlay">
  <a class="close"></a>
  <iframe class="contentWrap" style="width: 100%; height: 500px"></iframe>
 </div>
 <div style="display: none; position: absolute;" id="calroot">
  <div id="calhead">
   <a id="calprev"></a>
   <div id="caltitle"></div>
   <a id="calnext"></a>
  </div>
  <div id="calbody">
   <div id="caldays">
    <span>Sun</span><span>Mon</span><span>Tue</span><span>Wed</span><span>Thu</span><span>Fri</span><span>Sat</span>
   </div>
   <div id="calweeks">
    <!--   -->
   </div>
  </div>
 </div>
</body>
\end{lstlisting}



\subsubsection{Навігаційна панель}
\par На навігаційній панелі розташовані елементи швидкого доступу до завдань та задач.
З легкістю можна додати будь-яке завдання, при чому вибрати заголовок завдання, детальний опис та кінцевий час виконання (рисунок \ref{pic:page_navigation_new_task});
\begin{figure}[!ht]
\centering
    \includegraphics[width=0.3\textwidth]{page_navigation_new_task.png}
    \captionof{figure}{Приклад створення завдання із навігаційної панелі}\label{pic:page_navigation_new_task}
\end{figure}
\par Для зручності навігаційна панель рухається разом із прокруткою сторінки, тобто якщо навіть користувач перейде в низ сторінки, то йому панель буде завжди доступна -- це зроблено для простоти і швидкості доступу до створення нової нотатки та завдання.
\par Справа на навігаційній панелі розташовано меню користувача. Тут знаходяться кнопки переходу на профіль робочого та кнопка виходу із сайту (рисунок \ref{pic:page_navigation_profile}). Після виходу всі дані, які збережені в сесії будуть видалені.
\begin{figure}[!ht]
\centering
    \includegraphics[width=0.3\textwidth]{page_navigation_profile.png}
    \captionof{figure}{Панель користувача}\label{pic:page_navigation_profile}
\end{figure}


\subsubsection{Головне меню}
\par Навігація по веб ресурсу реалізована за допомогою головного меню. В головному меню відображаються всі доступні на сайті навігаційні посилання:
\begin{itemize}
  \item головна сторінка;
  \item корпоративна пошта;
  \item завдання;
  \item календар;
  \item нотатки;
  \item документи;
  \item список робочих;
  \item корпоративна вікі;
  \item блог.
\end{itemize}
\par При переході на будь-яке меню, воно зразу підсвічується -- це зроблено для зручності користувачеві, щоб було зразу видно де він знаходиться в даний момент часу. Програмно це відбувається за допомогою передачі з контролера в модель атрибута із назвою меню:
\begin{lstlisting}[language=Java]
uiModel.addAttribute("menu", "NOTE");
\end{lstlisting}
\par Потім в JSP вигляді головного меню відбувається перевірка на значення поточного меню, і якщо воно сходиться із атрибутом <<menu>> то додається css клас <<active>> (рисунок \ref{pic:page_menu}):

\begin{lstlisting}
<c:choose>
<c:when test="${menu eq 'NOTE' }"><li class="active"><a class="nav-icon icon-note" href="/ccms/notes">Notes</a></li></c:when>
<c:otherwise><li><a class="nav-icon icon-note" href="/ccms/notes">Notes</a></li></c:otherwise>
</c:choose>
\end{lstlisting}

\begin{figure}[!ht]
\centering
    \includegraphics[width=0.30\textwidth]{page_menu.png}
    \captionof{figure}{Навігаційне меню порталу}\label{pic:page_menu}
\end{figure}
\par Відповідно до переходу на певний пункт, відбувається запит контролеру MVC, і вибірка даних із бази даних через контролер з подальшою передачею на вигляд. Для прикладу запит для запису в БД та перевірка на валідність даних:
\begin{lstlisting}
@RequestMapping(method = RequestMethod.POST, produces = "text/html")
public String create(@Valid Note note, BindingResult bindingResult, Model uiModel, HttpServletRequest httpServletRequest) {
    if (bindingResult.hasErrors()) {
        populateEditForm(uiModel, note);
        uiModel.addAttribute("menu", "NOTE");
        return "redirect:/notes";
    }
    uiModel.asMap().clear();
    note.setAuthor(Worker.getPrincipal());
    note.setDatetime(new Date());
    note.persist();
    uiModel.addAttribute("menu", "NOTE");
    return "redirect:/notes";
}
\end{lstlisting}


\subsection{Робота із даними}
\par Для маппінгу даних із форми до бази даних використовується JPA із Hibernate фреймворком поверх нього. Для кожної форми створюється певний домен (по своїй суті persistence bean), котрий за допомогою анотацій із пакету Javax дає змогу переносити об'єкти Java в базу даних (за допомогою використання мови запитів Hibernate Query Language). Для прикладу bean для запису нотаток в базу даних:
\begin{lstlisting} 
@Configurable
@Entity
public class Note {

@NotNull
private String title;

@NotNull
@Size(max = 1000000)
private String text;

@Temporal(TemporalType.TIMESTAMP)
@DateTimeFormat(style = "M-")
private Date datetime;

@ManyToOne
private Worker author;

public static TypedQuery<Note> findNotesByAuthorEquals(Worker author) {
    if (author == null)
        throw new IllegalArgumentException("The author argument is required");
    EntityManager em = Note.entityManager();
    TypedQuery<Note> q = em.createQuery("SELECT o FROM Note AS o WHERE o.author = :author ORDER by o.id DESC", Note.class);
    q.setParameter("author", author);
    return q;
}

public String getTitle() {
    return this.title;
}

public void setTitle(String title) {
    this.title = title;
}
\end{lstlisting}
\par В вище наведеному коді кожне поле має свій метод на getter та setter, що дає змогу в вибірки даних, та статичний метод <<findNotesByAuthorEquals>> для пошуку повідомлень даного автора, що дає змогу в любому місці здійснити операції щодо пошуку цих повідомлень.

\subsection{Категорії порталу}
\subsubsection{Авторизація}
\par Сторінка авторизації пропонує користувачеві ввести свій логін та пароль (рисунок \ref{pic:page_login}) та у випадку неправильних даних буде відображена помилка (рисунок \ref{pic:page_login_error})

\begin{figure}[!ht]
\centering
    \includegraphics[width=0.50\textwidth]{page_login.png}
    \captionof{figure}{Форма авторизації користувача}\label{pic:page_login}
\end{figure}

\begin{figure}[!ht]
\centering
    \includegraphics[width=0.50\textwidth]{page_login_error.png}
    \captionof{figure}{Помилка авторизації користувача}\label{pic:page_login_error}
\end{figure}

\par Якщо введено вірні дані, то відбувається запис нової сесії в пам'ять та перенаправлення користувача на головну сторінку, або на сторінку із якої прийшов користувач.

\subsubsection{Корпоративна пошта}
\par Всі отримані листи користувач може переглянути в пункті пошта. Напроти меню <<пошта>> відображається кількість не прочитаних повідомлень (рисунок \ref{pic:page_mail}).
  \begin{figure}[!ht]
  \centering
      \includegraphics[width=1\textwidth]{page_mail.png}
      \captionof{figure}{Сторінка зі списком повідомлень}\label{pic:page_mail}
  \end{figure}
\par Над кожним повідомленням показано тему повідомлення. Головний текст повідомлення обрізаний, проте коли клікнути на повідомлення, збоку висунеться панель із повним описом повідомлення. Також під повідомленням показано час відправлення повідомлення та відправник повідомлення. При кліку на відправника, відбудеться перехід на його персональну сторінку. Кожне не прочитане повідомлення виділяється сирім кольором, для того щоб легше було його знайти, і при детальному перегляді його, колір забереться, і кількість повідомлень, що показуються біля меню -- буде зменшено.
\par Відправлення повідомлення можливе із персональної сторінки кожного користувача. Після переходу на персональну сторінку, слід надрукувати тему повідомлення та саме повідомлення (рисунок \ref{pic:page_send_message}).
  \begin{figure}[!ht]
  \centering
      \includegraphics[width=0.8\textwidth]{page_send_message.png}
      \captionof{figure}{Форма для відправлення повідомлень}\label{pic:page_send_message}
  \end{figure}
\par Зразу також доступний wysiwyg редактор і live перегляд повідомлення яке друкується. Доставка повідомлення відбувається моментально, адже використовується локальний сервер бази даних.

\subsubsection{Календар}
Персональний календар дає змогу показати всі занесені до нього нотатки та записи. Перегляд даних можливий у трьох проміжних режимах: на місяць, на тиждень (рисунок \ref{pic:page_calendar}) та на день.
  \begin{figure}[!ht]
  \centering
      \includegraphics[width=1\textwidth]{page_calendar.png}
      \captionof{figure}{Завдання на тиждень}\label{pic:page_calendar}
  \end{figure}
\par Завдання можуть бути додані як на певний проміжний період, так і на цілий день. Для маніпуляції записів в календарі використана технологія drag \& drop від jQuery.

\subsubsection{Завдання і задачі}
\par Категорія задач створена для збереження своїх задач (\ref{pic:page_navigation_new_task}) і можливістю їх перегляду в майбутньому. Це дає змогу всі свої важливі завдання тримати в одному місці (\ref{pic:page_task}).
  \begin{figure}[!ht]
  \centering
      \includegraphics[width=0.7\textwidth]{page_task.png}
      \captionof{figure}{Поточні завдання}\label{pic:page_task}
  \end{figure}
\par Кожне завдання яке не виконано ще, позначається аналогічно до непрочитаного повідомлення -- сірим кольором, це дає змогу зразу побачити всі поточні завдання. Біля кожного завдання вказано хто створив дане завдання та кінцевий час його виконання. Також в головному меню навпроти пункту <<завдання>> вказується кількість невиконаних на даний момент завдань. Також кожне завдання може бути позначене як виконане або ж невиконане.

\subsubsection{Користувачі}
\par Список всіх користувачів відображається в таблиці із деяким набором полів. Для переглядаючого доступні певні маніпуляції зі списком, такі як сортування та посторінкова навігація (рисунок \ref{pic:page_workers}). А у випадку, якщо користувач наділений правами адміністратора -- то має право на створення нового користувача (рисунок \ref{pic:page_create_worker}).

  \begin{figure}[!ht]
  \centering
      \includegraphics[width=1\textwidth]{page_workers.png}
      \captionof{figure}{Список користувачів із можливістю сортування}\label{pic:page_workers}
  \end{figure}

  \begin{figure}[!ht]
  \centering
      \includegraphics[width=0.7\textwidth]{page_create_worker.png}
      \captionof{figure}{Створення нового користувача}\label{pic:page_create_worker}
  \end{figure}

\par Після переходу на сторінку користувача, у його профайлі буде відображена вся детально інформація (рисунок \ref{pic:page_worker})
  \begin{figure}[!ht]
  \centering
      \includegraphics[width=1\textwidth]{page_worker.png}
      \captionof{figure}{Профайл користувача}\label{pic:page_worker}
  \end{figure}
\par Якщо при введенні не валідних даних, або ж залишити незаповненим обов'язкове поле -- то буде повідомлена відповідна помилка, і дані не потраплять на перевірку на сервер. Якщо ж зловмиснику вдасться все ж таки обійти перевірку форми, то сам сервер не пустить додати не валідні дані до бази даних, оскільки всі дані перевіряються другий раз за допомогою binding result об'єкта та анотації @Valid:
\begin{lstlisting}
public String update(@Valid Worker worker, BindingResult bindingResult, Model uiModel, HttpServletRequest httpServletRequest) {
if (bindingResult.hasErrors()) {
    populateEditForm(uiModel, worker);
    uiModel.addAttribute("menu", "WORKER");
    return "workers/update";
}
\end{lstlisting}
\par Якщо після проходження валідації є допущені помилка то дані назад <<повертаються>> на форму і відображається помилка. В іншому випадку, дані передадуться на модель та відбудеться запис у базу даних
\begin{lstlisting}
uiModel.asMap().clear();
worker.merge();
\end{lstlisting}


\subsubsection{Документи}
Дана категорія призначена для зберігання документів для їх спільного використання, для прикладу це можуть бути презентації, облікові документи чи просто інші нотатки. Для додавання доступні три категорії: презентації, текстові документи та таблиці. При завантаженні нового документа на портал, слід вказати категорію в котру повинен попасти документ, та вказати чи документ призначений для загально використання чи тільки для персонального.
\par Всі загальні документи доступні для завантаження та можливістю подальшого перегляду.

\subsubsection{Корпоративна wiki}
\par Корпоративна wiki в основному призначена для розповсюдження цікавої інформації між користувачами та являє собою єдине центральне сховище з можливість будь-якої маніпуляції документами. Кожна стаття має свою певну категорію -- що спрощує подальшу навігацію та пошук.

\subsubsection{Корпоративний блог}
\par Корпоративний блог має спільні риси із wiki, проте додати інформацію в нього тільки має право адміністратор та наділені такими правами групи користувачів. Основна ціль блогу -- це швидше інформування робочих про новини порталу та компанії.

\section{ЕКОНОМІЧНА ДОЦІЛЬНІСТЬ ВИКОРИСТАННЯ ПРОГРАМНОГО ЗАБЕЗПЕЧЕННЯ}

\subsection{Економічна доцільність розробки програмного забезпечення та його впровадження}

В даному проекті необхідно реалізувати корпоративну систему для спільної і одночасної роботи працівників деякої компанії. В ньому буде реалізовано систему обміну повідомленнями, управління задачами і завданнями, зручне ведення корпоративного календаря, спільна робота над документами різного типу (текстові документи, презентації тощо), система корпоративної вікі та блог. 
\par Як відомо, кожний продукт, який розробляється сьогодні з подальшим впровадженням на ринок потребує обґрунтування з економічної точки зору, а саме доцільності даного продукту. Дане обґрунтування необхідне для того, щоб вчасно припинити (при втраті актуальності або надмірних витратах) розробку або здійснити необхідні інвестування в проект для забезпечення необхідними програмними або апаратними засобами розробників з метою одержання очікуваних результатів. Економічний ефект розробленого продукту визначається на основі економічних показників, які дають можливість прогнозувати результат від впровадження даного програмного продукту.
\par Існує багато методів визначення економічних показників доцільності впровадження та використання будь якого програмного продукту. Враховуючи інтенсивне впровадження комп’ютерної техніки в корпоративній сфері, на сьогодні такий аналіз є невід’ємною частиною попереднього аналізу аналогічних робіт, оскільки саме результат економічних показників доцільності дозволяє визначити доцільність розробки програмного продукту.
\par В даній роботі проводиться розрахунок економічних показників та аналіз всієї роботи по розробці корпоративної системи.

\subsection{Побудова мережевого графа}
Мережевий граф є основним плановим документом в системі мережевого планування і керування, що являє собою інформаційно-динамічну модель, в якій зображуються взаємозв'язки і результати всіх робіт, необхідних для досягнення кінцевої мети розробки, тобто мережевий граф - це наочне відображення плану робіт.
\par В мережевому графі детально чи укрупнено показано, що, в якій послідовності, коли, за який час, для чого необхідно виконати, щоб забезпечити закінчення всіх робіт не пізніше заданого, директивного терміну.
\par Порядок побудови мережевих графів визначається прийнятою технологією і організацією робіт. Мережеві графи тільки відображають існуючу або проектовану черговість і взаємозв'язок виконання робіт.
\par По кожній роботі необхідно враховувати:
\begin{enumerate}
	\item які роботи повинні бути завершені раніше, ніж почнеться дана робота;
	\item які роботи можуть початись після завершення даної роботи;
	\item які інші роботи повинні виконуватись одночасно з виконуванням даної роботи.
\end{enumerate}
\par Аналізуючи мережевий граф можна виділити його головні елементи: події і роботи. Розглянемо детальніше значення термінів:
\begin{enumerate}
	\item подія - це стан, момент досягнення проміжної або кінцевої цілі розробки.
	\item робота - це розтягнений в часі процес, необхідний для здійснення події. Кожна робота має попередню подію і закінчується визначеною подією.
\end{enumerate}

\par На мережевих графах подія відображається колом, а робота --- стрілкою. До основних параметрів мережевого графа відносяться: критичний шлях, резерви часу подій. Ці параметри є вихідними для одержання ряду додаткових характеристик, а також для аналізу мережі чи для аналізу складеного плану розробки.
\par Резерв часу події - це такий проміжок часу, на який може бути відкладене здійснення цієї події без порушення термінів завершення розробки в цілому. Резерви часу існують в мережевому графі в усіх випадках, коли існує більш ніж один шлях різної тривалості.
\par Резерв часу події К визначається як різниця між пізнім $T_{p}$ і раннім $T_{r}$ термінами завершення події за формулою

\begin{equation}
	K=\frac{T_{p}}{T_{r}}
\end{equation}

\par Найбільш пізній з допустимих термінів $T_{p}$ - це такий термін здійснення події, перевищення якого викличе аналогічну затримку завершальної події. Іншими словами, якщо подія наступила в момент $T_{p}$, вона потрапила в критичну зону і наступні за нею роботи повинні знаходитись під таким же контролем як і роботи критичного шляху.

\par Найбільш ранній з можливих термінів здійснення події $T_{p}$ --- це термін необхідний для виконання всіх робіт, що передують цій події. Цей час знаходиться шляхом вибору максимального значення із тривалості всіх шляхів, що приводять до даної події.
\par Вихідні дані мережевого графа представлені в таблицях \ref{t:eco_1} та \ref{t:eco_2}.


{\footnotesize
\begin{longtable}{|c|c|}

\captionsetup{justification=centering}
\caption{Події мережевого графа}\label{t:eco_1}\\
\hline
\multicolumn{1}{|c|}{\textbf{№ події}}&
\multicolumn{1}{c|}{\textbf{Подія}}\\\hline

\endfirsthead
\caption*{\hfill Продовження таблиці \ref{t:eco_1}}\\\hline

\multicolumn{1}{|c|}{\textbf{№ події}}&
\multicolumn{1}{c|}{\textbf{Подія}}\\\hline
\endhead

0 & Отримання завдання на дипломне проектування\\ \hline
1 & Аналіз проблеми дипломного проектування \\ \hline
2 & Ознайомлення з літературою на задану тему \\ \hline
3 & Пошук інформації в мережі INTERNET \\ \hline
4 & Підбір необхідних джерел інформації  \\ \hline
5 & Аналіз підібраного матеріалу  \\ \hline
6 & Визначення задач, які виникають при розробці  \\ \hline
7 & Розгляд існуючих способів розробки корпоративних систем \\ \hline
8 & Аналіз існуючих способів розробки  \\ \hline
9 & Пошук існуючих корпоративних систем \\ \hline
10 & Аналіз знайдених аналогів та їх функціональності \\ \hline
11 & Розробка структури алгоритму  \\ \hline
12 & Розробка алгоритму програми  \\ \hline
13 & Вибір серверної  платформи для реалізації завдання  \\ \hline
14 & Визначення основних та допоміжних програмних модулів  \\ \hline
15 & Реалізація програмних модулів в середовищі програмування  \\ \hline
16 & Попереднє налагодження програмних модулів \\ \hline
17 & Остаточне налагодження програми \\ \hline
18 & Тестування програмного продукту \\ \hline
19 & Визначення економічної доцільності використання програми \\ \hline
20 & Завершення роботи АБВГ\\ \hline

\end{longtable}
}


{\footnotesize
\begin{longtable}{|c|c|c|}
\captionsetup{justification=centering}
\caption{Роботи мережевого графа}\label{t:eco_2}\\
\hline
\multicolumn{1}{|c|}{\textbf{Номери робіт}}&
\multicolumn{1}{c|}{\textbf{Роботи}}&
\multicolumn{1}{c|}{\textbf{Тривалість, \newline дні}}\\\hline

\endfirsthead
\caption*{\hfill Продовження таблиці \ref{t:eco_2}}\\\hline

\multicolumn{1}{|c|}{\textbf{Номери робіт}}&
\multicolumn{1}{c|}{\textbf{Роботи}}&
\multicolumn{1}{c|}{\textbf{Тривалість, дні}}\\\hline
\endhead

0-1 & Аналіз завдання дипломного проекту & 2\\ \hline
1-2 & Огляд літератури & 3\\ \hline
1-3 & Огляд інформації в INTERNET & 3\\ \hline
3-4 & Робота з підібраним матеріалом з INTERNET & 4\\ \hline
2-4 & Робота з підібраним технічним матеріалом & 3\\ \hline
4-5 & Аналіз вимог до системи та її функціональності & 4\\ \hline
5-6 & Виділення та групування задач розробки & 3\\ \hline
6-7 & Пошук та розгляд існуючих методів реалізації & 7\\ \hline
7-8 & Аналіз та компонування існуючих способів розробки & 4\\ \hline
8-9 & Пошук аналогів розробленої системи & 5\\ \hline
8-10 & Аналіз аналогів розробленої системи & 4\\ \hline
9-11 & Завершення аналізу аналогів та вибір способу реалізації & 2\\ \hline
11-13 & Розробка структури алгоритму & 5\\ \hline
10-12 & Складання алгоритму програми та його аналіз & 2\\ \hline
12-13 & Розробка структури програми & 7\\ \hline
13-14 & Уточнення виду вхідних даних для програми & 5\\ \hline
14-15 & Аналіз інструментальних засобів створення програми & 2\\ \hline
15-16 & Підбір середовища програмування & 3\\ \hline
16-17 & Написання коду модулів програми & 14\\ \hline
17-18 & Налагодження всіх модулів програми & 7\\ \hline
18-19 & Завершення етапу налагодження програми & 3\\ \hline
19-20 & Тест програми та аналіз результатів тестування & 2\\ \hline
20-21 & Аналіз економічних показників & 5\\ \hline
21-22 & Завершення роботи & 14\\ \hline

\end{longtable}
}


\begin{center}
		\includegraphics[width=1.00\textwidth]{ecomonic_graph.png}
		\captionof{figure}{Мережевий граф виконаних робіт}\label{t:eco_graph}
\end{center}

\par На рисунку \ref{t:eco_graph} зображений мережевий граф, який отримано із вихідних даних таблиць. Знаходимо критичний шлях і розраховуємо ранній, пізній час і резерв часу.
\par Критичний шлях --- це найбільш тривала по часу послідовність робіт, які ведуть від вихідної до завершальної події. Величина критичного шляху визначає термін виконання всього комплексу по плануванню робіт.
\par Зміна тривалості будь-якої роботи, що лежить на критичному шляху, відповідним чином змінює термін настання завершальної події, тобто дату досягнення кінцевої мети, яка ставиться при плануванні розробки.
\par При плануванні комплексу операцій критичний шлях дозволяє знайти термін настання завершальної події. В процесі керування ходом розробки увага керівництва зосереджується на роботах критичного шляху. Це дозволяє найбільш  доцільно   і   оперативно  контролювати   обмежене  число  робіт, що впливають на термін розробки, а також краще використати існуючі ресурси.
\par Оскільки в даному випадку мережевий граф досить простий, очевидно що критичний шлях рівний 95.
\par Дані розрахунків часу подій приведені в таблиці \ref{t:eco_3}.


{\footnotesize
\begin{longtable}{|c|c|c|c|}

\captionsetup{justification=centering}
\caption{Параметри подій мережевого графіка}\label{t:eco_3}\\
\hline
\multicolumn{1}{|c|}{\textbf{№ події}}&
\multicolumn{1}{c|}{\textbf{Ранній час}}&
\multicolumn{1}{c|}{\textbf{Пізній час}}&
\multicolumn{1}{c|}{\textbf{Резерв часу}}\\\hline

\endfirsthead
\caption*{\hfill Продовження таблиці \ref{t:eco_3}}\\\hline

\multicolumn{1}{|c|}{\textbf{№ події}}&
\multicolumn{1}{c|}{\textbf{Ранній час}}&
\multicolumn{1}{c|}{\textbf{Пізній час}}&
\multicolumn{1}{c|}{\textbf{Резерв часу}}\\\hline
\endhead

0 & 0 & 0 & 0\\ \hline
1 & 2 & 2 & 0\\ \hline
2 & 5 & 6 & 1\\ \hline
3 & 5 & 5 & 0\\ \hline
4 & 9 & 9 & 0\\ \hline
5 & 13 & 13 & 0\\ \hline
6 & 16 & 16 & 0\\ \hline
7 & 23 & 23 & 0\\ \hline
8 & 27 & 27 & 0\\ \hline
9 & 32 & 32 & 0\\ \hline
10 & 31 & 31 & 0\\ \hline
11 & 34 & 35 & 1\\ \hline
12 & 33 & 33 & 0\\ \hline
13 & 40 & 40 & 0\\ \hline
14 & 45 & 45 & 0\\ \hline
15 & 47 & 47 & 0\\ \hline
16 & 50 & 50 & 0\\ \hline
17 & 64 & 64 & 0\\ \hline
18 & 71 & 71 & 0\\ \hline
19 & 74 & 74 & 0\\ \hline
20 & 76 & 76 & 0\\ \hline
21 & 81 & 81 & 0\\ \hline
22 & 95 & 95 & 0\\ \hline

\end{longtable}}

\subsection{Економічне обґрунтування розробки та впровадження програми}
Економічне обґрунтування розробки та впровадження програми будемо здійснювати на аналізі таких економічних показників:
\par $S_{po}$ -- сумарні витрати на розробку програмного забезпечення;
\par $\Delta{E_{e2/1}}$ -- експлуатаційні витрати.
\par Розрахунок відповідних коефіцієнтів проводиться з врахуванням того, що варіаційні задачі діагностування раніше виконувались вручну.

\subsubsection{Розрахунок витрат на розробку програмного забезпечення}
Сумарні витрати на розробку програмного забезпечення $S_{po}$ визначаються за формулою:
\begin{equation}
	S_{po} = \sum_{i}t_{po_{i}}\cdot{B_{po_{i}}}\cdot{[(1+\omega_{d})\cdot{(1+\omega_{c})}+\omega_{n}]}+t_{mo}\cdot{e_{g}},
\end{equation}
\par де $t_{po_{i}}$ -- час, що витрачається на розробку даної програми працівником $i-$ої кваліфікації, люд.-міс;
\par $B_{po_{i}}$ -- основна заробітна плата розробника $i-$-ої кваліфікації, грн/міс;
\par $\omega_{d}$ -- коефіцієнт, що враховує додаткову заробітну плату розробникам програми, у відсотках від основної заробітної плати;
\par $\omega_{c}$ -- коефіцієнт, що враховує нарахування органам соціального захисту на заробітну плату, у відсотках від основної та додаткової заробітної плати;
\par $\omega_{n}$ -- коефіцієнт, що враховує накладні витрати установи, в якій розробляється ця програма, у відсотках до основної заробітної плати розробника;
\par $t_{mo}$ -- машинний час ЕОМ, необхідний для налагоджування даної програми, машино-год;
\par $e_{g}$ -- експлуатаційні витрати, що припадають на 1 год машинного часу.

\par Значення коефіцієнтів $\omega_{d}=0$; $\omega_{c}=0.375$; $\omega_{n}=0.42$. Нехай $t_{mo}$ = 1 люд.-міс, а $B_{po_{i}}$ = 3000 грн. Експлуатаційні витрати, що припадають на 1 год машинного часу, можуть бути визначені за витратою електроенергії:

\begin{equation}\label{eq:eco_eq_3}
	S_{g} = P_{cp}\cdot{C_{bod}},
\end{equation}
\par де $P_{cp}$ = 90 Вт -- споживана потужність ЕОМ (ноутбук);
\par $C_{bod}$ = 0.8762 -- вартість 1 кВт/год електроенергії для підприємств.

\par Отже, за \eqref{eq:eco_eq_3}:
\begin{center}
	\center{$e_{g} = 0.09\cdot{0.8762}=0,079$ грн/год.}
\end{center}

\par Необхідний час налагодження програми становить 24 машино-год.
\par Сумарні витрати на розробку програмного забезпечення складуть:
	\par
\begin{center}
	$S_{po} = 1\cdot3000\cdot((1+0)\cdot(1+0.375)+0.42)+24\cdot0.079=5386.90$ грн.
\end{center}
\par Використання запропонованої програми не потребує додаткових капітальних вкладень у користувача.


\subsubsection{Розрахунок можливого прибутку}
\par Даний продукт буде розповсюджуватися на ліцензією GNU General Public License, що означає безкоштовне її розповсюдження. Тому для того щоб повернутися витрачені кошти на її розробку і підтримку, варто використовувати загальні методи поширення open source програм, це: заробіток підтримки користувачів продукту (супорт). 
\par Буде введено два тарифи: річна підписка (2000 грн.), та помісячна (200 грн.).
\par Прогнози, зроблені на основі дослідження ринку, дозволяють нам очікувати наступний прибуток за 1 рік підтримки користувачів продукту на ринку.
\par Середня кількість компаній за рік буде становити порядку 10-ти. В середньому на ринку, кожна друга компанія буде користуватися послугою супорту і налаштування продукту. Решта половина буде тільки використовувати разову місячну передплату. Отже очікуваний прибуток за 1 рік на ринку буде становити:
\par 5 місячний передплат -- $5\cdot200 = 1000$ грн.
\par 5 річних передплат -- $5\cdot2000 = 10000$ грн.
\par Очікуваний прибуток за рік становитиме: 
\par $P = (10000+1000)-5386.90 = 5613.1$ грн.
\par Чистий прибуток: $P_{ch.} = (1-0.21)\cdot5613.1=4434.35$ грн.
\par Чистий місячний прибуток буде становити: $P_{m.ch.}=369.53$ грн.


\subsubsection{Розрахунок зведених економічних показників}
\par Термін   окупності   додаткових   капітальних   вкладень   визначається   за формулою:
\begin{equation}\label{eq:eco_eq_4}
	T_{OK} = \frac{S_{po}}{P_{ch.}}
\end{equation}
\par Отже, за \eqref{eq:eco_eq_4}
\begin{center}
	$T_{OK}=5386.90/369.53=14.5$ місяця.
\end{center}

\par Ефект, який отримує корпорація при користуванні даним продуктом полягає у легкості і гнучкості взаємодії між користувачами, спільною роботу над документами і завданнями.
\par В таблиці \ref{t:eco_4} наведені зведені економічні показники системи. З вище наведених розрахунків видно, що розробка та впровадження даної програми є економічно доцільною. 


{\footnotesize
\begin{longtable}{|c|c|c|c|}

\captionsetup{justification=centering}
\caption{Зведені економічні показники розробки системи}\label{t:eco_4}\\
\hline
\multicolumn{1}{|c|}{\textbf{Показник}}&
\multicolumn{1}{c|}{\textbf{Розмірність}}&
\multicolumn{1}{c|}{\textbf{Значення}}\\\hline

\endfirsthead
\caption*{\hfill Продовження таблиці \ref{t:eco_4}}\\\hline

\multicolumn{1}{|c|}{\textbf{Показник}}&
\multicolumn{1}{c|}{\textbf{Розмірність}}&
\multicolumn{1}{c|}{\textbf{Значення}}\\\hline
\endhead

Витрати на розробку програмного забезпечення & грн & 5386.90 \\ \hline
Очікуваний економічний ефект (за рік) & грн & 4434.35 \\ \hline
Термін окупності розробки графічного редактора & місяць & 14.5 \\ \hline

\end{longtable}
}

\par Таким чином, з цих економічних розрахунків випливає, що розробка корпоративної системи, розповсюдження якої базується на ліцензії GNU є економічно доцільним і дозволяє отримувати прибутки від підтримки користувачів і налаштування ПЗ.


\section*{ВИСНОВОК}
\addcontentsline{toc}{section}{ВИСНОВОК}
Завдяки сучасним технологіям і корпоративним стандартам, розвиток розробки комерційних продуктів виріс дуже стрімко. 
Зокрема сюди і відноситься відносно молодий напрямок --- це розробка корпоративних порталів. 
Було встановлено стандарти щодо розробки додатків і аплікацій, це помогло добитися легкої інтеграції і взаємодії. 
Також проведено аналіз сучасного стану і потреб ринку в даній сфері, наведено всі вимоги до програмного продукту.



%%% Bibliography
\renewcommand{\refname}{ВИКОРИСТАНІ ДЖЕРЕЛА}
\begin{thebibliography}{99}
\bibitem{portlet_2} http://www.jcp.org/en/jsr/detail?id=286 - стандарт портлетів Java Portlet 2.0 Standard 
\bibitem{portlet_1} http://www.jcp.org/en/jsr/detail?id=168 - стандарт портлетів Java Portlet 1.0 Standard 
\bibitem{google} http://google.com - пошук доступної в інтернеті інформації
\bibitem{servlet_api} http://tomcat.apache.org/tomcat-5.5-doc/servletapi/javax/servlet/Servlet.html - специфікація серлетів
\bibitem{pz_nung_edu_ua} http://pz.nung.edu.ua/ - сайт кафедри ПЗАС
\bibitem{intranetno} http://www.intranetno.ru/ - бізнес рішення на базі SaaS, PaaS
\end{thebibliography}


\end{document}

