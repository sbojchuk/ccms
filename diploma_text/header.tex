%%% initial
\documentclass[ukrainian,14pt,utf8,
%pointsection, % це хз для чого
nocolumnsxix,
nocolumnxxxi,
nocolumnxxxii,
floatsection, %підпис плаваючих об'єктів згдіно номера розліду (таблиці, рисунки)
hpadding=5mm, %відстань до рамки зліва, справа
vpadding=5mm %відстань до рамки зліва, справа
]{eskdtext}

%%% for working with math
%\usepackage{amsmath}
%\usepackage{amssymb}

%%% for working some LaTeX packages
\usepackage{xecyr}

%%% for fonts
\usepackage{xltxtra}

%%% use for formula counting for every section
\numberwithin{equation}{section} 

%%% Times New Roman - main font
\setmainfont[Mapping=tex-text]{Times New Roman}
%%% Courier New - fot math
\setmonofont[Scale=MatchLowercase]{Courier New}

%%% for ---, --, << >> и т.п.
\defaultfontfeatures{Mapping=tex-text}

%%% ukrainian text
\usepackage{polyglossia}
\setdefaultlanguage{ukrainian}
\newfontfamily\ukrainianfont{Times New Roman}

%%% for words wrapping
%\XeTeXinterchartokenstate=1
%\XeTeXcharclass `\- 24
%\XeTeXinterchartoks 24 0 ={\hskip\z@skip}
%\XeTeXinterchartoks 0 24 ={\nobreak}

% для абзацу %
\usepackage{indentfirst}

% code highlighting %
\usepackage{listings}
\lstset{
language=Java,
basicstyle=\footnotesize\sffamily,
numbers=left,
numberstyle=\tiny,
frame=tb,
columns=fullflexible,
showstringspaces=false,
breaklines=true 
}

% automaticaly wrap to other line
\sloppy
\usepackage{setspace}
%\onehalfspacing
\linespread{1.5}

%%% for graphix
\usepackage{graphicx}
\graphicspath{{images/}} %path to images
%%% titile for pictures
\addto{\captionsukrainian}{\renewcommand{\figurename}{Рисунок}}


%%% for internet ulrs
\usepackage{url}

%%% subtitiles(\subsubsection) don't show in article
%\setcounter{tocdepth}{2}

%%% new section from new page
\let\stdsection\section
\renewcommand\section{\newpage\stdsection}

%%% плаваючі обєкти підпис
%\renewcommand\section{\newpage\stdsection}

%%% Numering subtopics begin with B (B.1)
%\makeatletter
%\renewcommand\thesubsection{\ifnum\c@section=0{В.\arabic{subsection}}\else{\arabic{section}.\arabic{subsection}}\fi}
%\makeatother

%%% for coloring rows
\usepackage[table]{xcolor}

%%% для розриву таблиць 
\usepackage{longtable}





