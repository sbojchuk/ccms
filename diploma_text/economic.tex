\section{ЕКОНОМІЧНА ДОЦІЛЬНІСТЬ ВИКОРИСТАННЯ ПРОГРАМНОГО ЗАБЕЗПЕЧЕННЯ}

\subsection{Економічна доцільність розробки програмного забезпечення та його впровадження}

В даному проекті необхідно реалізувати корпоративну систему для спільної і одночасної роботи працівників деякої компанії. В ньому буде реалізовано систему обміну повідомленнями, управління задачами і завданнями, зручне ведення корпоративного календаря, спільна робота над документами різного типу (текстові документи, презентації тощо), система корпоративної вікі та блог. 
\par Як відомо, кожний продукт, який розробляється сьогодні з подальшим впровадженням на ринок потребує обґрунтування з економічної точки зору, а саме доцільності даного продукту. Дане обґрунтування необхідне для того, щоб вчасно припинити (при втраті актуальності або надмірних витратах) розробку або здійснити необхідні інвестування в проект для забезпечення необхідними програмними або апаратними засобами розробників з метою одержання очікуваних результатів. Економічний ефект розробленого продукту визначається на основі економічних показників, які дають можливість прогнозувати результат від впровадження даного програмного продукту.
\par Існує багато методів визначення економічних показників доцільності впровадження та використання будь якого програмного продукту. Враховуючи інтенсивне впровадження комп’ютерної техніки в корпоративній сфері, на сьогодні такий аналіз є невід’ємною частиною попереднього аналізу аналогічних робіт, оскільки саме результат економічних показників доцільності дозволяє визначити доцільність розробки програмного продукту.
\par В даній роботі проводиться розрахунок економічних показників та аналіз всієї роботи по розробці корпоративної системи.

\subsection{Побудова мережевого графа}
Мережевий граф є основним плановим документом в системі мережевого планування і керування, що являє собою інформаційно-динамічну модель, в якій зображуються взаємозв'язки і результати всіх робіт, необхідних для досягнення кінцевої мети розробки, тобто мережевий граф - це наочне відображення плану робіт.
\par В мережевому графі детально чи укрупнено показано, що, в якій послідовності, коли, за який час, для чого необхідно виконати, щоб забезпечити закінчення всіх робіт не пізніше заданого, директивного терміну.
\par Порядок побудови мережевих графів визначається прийнятою технологією і організацією робіт. Мережеві графи тільки відображають існуючу або проектовану черговість і взаємозв'язок виконання робіт.
\par По кожній роботі необхідно враховувати:
\begin{enumerate}
	\item які роботи повинні бути завершені раніше, ніж почнеться дана робота;
	\item які роботи можуть початись після завершення даної роботи;
	\item які інші роботи повинні виконуватись одночасно з виконуванням даної роботи.
\end{enumerate}

\par Аналізуючи мережевий граф можна виділити його головні елементи: події і роботи. Розглянемо детальніше значення термінів:
\begin{enumerate}
	\item подія - це стан, момент досягнення проміжної або кінцевої цілі розробки.
	\item робота - це розтягнений в часі процес, необхідний для здійснення події. Кожна робота має попередню подію і закінчується визначеною подією.
\end{enumerate}

\par На мережевих графах подія відображається колом, а робота --- стрілкою. До основних параметрів мережевого графа відносяться: критичний шлях, резерви часу подій. Ці параметри є вихідними для одержання ряду додаткових характеристик, а також для аналізу мережі чи для аналізу складеного плану розробки.
\par Резерв часу події - це такий проміжок часу, на який може бути відкладене здійснення цієї події без порушення термінів завершення розробки в цілому. Резерви часу існують в мережевому графі в усіх випадках, коли існує більш ніж один шлях різної тривалості.


<math>z=x+y</math>
$x=\frac{1+y}{1+2z^2}$

\begin{equation}
   \frac{1}{\sigma\sqrt{2\pi}}\exp\left(-\frac{(x-\mu)^2}{2\sigma^2}\right)
 \end{equation}